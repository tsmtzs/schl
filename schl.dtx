% \iffalse meta-comment
%
% Copyright (C) 2019 by Tasos Tsesmetzis
%
% This file may be distributed and/or modified under the
% conditions of the MIT License. 
%
% \fi

% \iffalse
%<package>\NeedsTeXFormat{LaTeX2e}
%<package>\ProvidesPackage{schl}
%<package>[2019/11/07 v1.0 Custom package]
%
% \iffalse meta-comment
%% %%%%%%%%%%%%%%%%%%%%%%%%%%%%%%%%%%%%%%%%%%%%%%%%%%%%%%%%%%%%%%%%%%%%%%%%%%%%%%%%%%%%%%%%%%%%
%% REQUIRED PACKAGES
%% %%%%%%%%%%%%%%%%%%%%%%%%%%%%%%%%%%%%%%%%%%%%%%%%%%%%%%%%%%%%%%%%%%%%%%%%%%%%%%%%%%%%%%%%%%%%
% \fi
%<package>\RequirePackage{fontspec}
%<package>\RequirePackage[shortlabels]{enumitem}
%<package>\RequirePackage{mdframed}
%<package>\RequirePackage{amsmath}
%<package>\RequirePackage{calc}
%<package>\RequirePackage{etoolbox}
%
% \iffalse meta-comment
%% 
%% %%%%%%%%%%%%%%%%%%%%%%%%%%%%%%%%%%%%%%%%%%%%%%%%%%%%%%%%%%%%%%%%%%%%%%%%%%%%%%%%%%%%%%%%%%%%
%% PACKAGE OPTIONS
%% %%%%%%%%%%%%%%%%%%%%%%%%%%%%%%%%%%%%%%%%%%%%%%%%%%%%%%%%%%%%%%%%%%%%%%%%%%%%%%%%%%%%%%%%%%%%
% \fi
%<schl-template>\ProvidesFile{languages/schl-template.def}
%
%<package>\DeclareOption{default}{
%<package>  %% %%%%%%%%%%%%%%%%%%%%%%%%%%%%%%%%%%%%%%%%%%%%%%%%%%%%%%%%%%%%%%%%%%%%%%%%%%%%%%%%%%%%%%%%%%
%% PACKAGE schl
%% UNDEFINED TERMS
%% This template file can be used for localizing the various terms of the schl package.
%% %%%%%%%%%%%%%%%%%%%%%%%%%%%%%%%%%%%%%%%%%%%%%%%%%%%%%%%%%%%%%%%%%%%%%%%%%%%%%%%%%%%%%%%%%%
\ProvidesFile{schl-template.def}[2019/09/03 v0.1 Term macros of the schl package]

\def\points@term#1{undefined}

% default wish
\def\schl@wish{undefined}

% \schl@authorityi > \schl@authorityii > \schl@authorityii
\def\schl@authorityi{undefined}
\def\schl@authorityii{undefined}
\def\schl@authorityiii{undefined}
\def\schl@school{undefined}
\def\schl@grade{undefined}
\def\schl@subject{undefined}
\def\schl@teacher{undefined}
\def\schl@schoolyear{undefined}
\def\schl@date{undefined}
\def\schl@examtime{undefined}

\def\grade@term{undefined}
\def\subject@term{undefined}
\def\testwriter@term{undefined}
\def\testsupervisor@term{undefined}
\def\date@term{undefined}
\def\trueabbr@term{undefined}
\def\falseabbr@term{undefined}

\def\schoolyear@term{undefined}
\def\schoolyearabbr@term{undefined}
\def\examperiod@term{undefined}
\def\lastname@term{undefined}
\def\name@term{undefined}
\def\examnoabbr@term{undefined}
\def\time@term{undefined}
\def\duration@term{undefined}

\def\fullname@term{undefined}

\def\deadline@term{undefined}
\def\remark@term{undefined}
\def\reminder@term{undefined}

\def\theoryheader@term{undefined}
\def\exerciseheader@term{undefined}

\def\worksheet@term{undefined}

\def\termtest@term{undefined}

\def\exams@term{undefined}
\def\period@term{undefined}

\def\headmaster@term{undefined}

\def\answerabbr@term{undefined}

\def\exercise@term{undefined}
\def\question@term{undefined}
\def\task@term{undefined}
\def\solution@term{undefined}
\def\hint@term{undefined}

\endinput
%<package>}
%
%<schl-greek>\ProvidesFile{languages/schl-greek.def}
%
%<package>\DeclareOption{greek}{
%<package>  %% %%%%%%%%%%%%%%%%%%%%%%%%%%%%%%%%%%%%%%%%%%%%%%%%%%%%%%%%%%%%%%%%%%%%%%%%%%%%%%%%%%%%%%%%%%
%% PACKAGE schl
%% TERMS IN GREEK
%% %%%%%%%%%%%%%%%%%%%%%%%%%%%%%%%%%%%%%%%%%%%%%%%%%%%%%%%%%%%%%%%%%%%%%%%%%%%%%%%%%%%%%%%%%%
\ProvidesFile{schl-greek.def}[2019/07/07 v0.1 Terms in Greek for the schl package]

\def\points@term#1{%
  \ifnum 1=#1
  μονάδα
  \else
  μονάδες
  \fi
}

% default wish
\def\schl@wish{\letterspace{\defaultletterspace}ΚΑΛΗ ΕΠΙΤΥΧΙΑ}

% \schl@authorityi > \schl@authorityii > \schl@authorityii
\def\schl@authorityi{\letterspace{\defaultletterspace}ΕΛΛΗΝΙΚΗ ΔΗΜΟΚΡΑΤΙΑ}
\def\schl@authorityii{\letterspace{\defaultletterspace}ΥΠΟΥΡΓΕΙΟ ΠΑΙΔΕΙΑΣ, ΕΡΕΥΝΑΣ \& ΘΡΗΣΚΕΥΜΑΤΩΝ}
\def\schl@authorityiii{ΔΕΝ ΕΧΕΙ ΟΡΙΣΤΕΙ}
\def\schl@school{ΔΕΝ ΕΧΕΙ ΟΡΙΣΤΕΙ}
\def\schl@grade{ΔΕΝ ΕΧΕΙ ΟΡΙΣΤΕΙ}
\def\schl@subject{ΔΕΝ ΕΧΕΙ ΟΡΙΣΤΕΙ}
\def\schl@teacher{ΔΕΝ ΕΧΕΙ ΟΡΙΣΤΕΙ}
\def\schl@schoolyear{ΔΕΝ ΕΧΕΙ ΟΡΙΣΤΕΙ}
\def\schl@schldate{ΔΕΝ ΕΧΕΙ ΟΡΙΣΤΕΙ}
\def\schl@examtime{ΔΕΝ ΕΧΕΙ ΟΡΙΣΤΕΙ}

\def\grade@term{Τάξη}
\def\subject@term{Μάθημα}
\def\testwriter@term{Εισηγητής}
\def\testsupervisor@term{Επιτηρητής}
\def\date@term{Ημερομηνία}
\def\trueabbr@term{Σ}
\def\falseabbr@term{Λ}

\def\schoolyear@term{ΣΧΟΛΙΚΟ ΕΤΟΣ}
\def\schoolyearabbr@term{ΣΧΟΛ\@.\, ΕΤΟΣ}
\def\examperiod@term{ΕΞΕΤΑΣΤΙΚΗ ΠΕΡΙΟΔΟΣ}
\def\lastname@term{ΕΠΙΘΕΤΟ}
\def\name@term{ΟΝΟΜΑ}
\def\examnoabbr@term{Α\@.Μ\@.Μ\.}
\def\time@term{ΩΡΑ}
\def\duration@term{Διάρκεια}

\def\fullname@term{Ονοματεπώνυμο}

\def\deadline@term{Παράδοση}
\def\remark@term{Παρατήρηση}

\def\theoryheader@term{ΘΕΩΡΙΑ}
\def\exerciseheader@term{ΑΣΚΗΣΕΙΣ}

\def\worksheet@term{Φύλλο εργασίας}

\def\termtest@term{Διαγώνισμα}

\def\exams@term{ΕΞΕΤΑΣΕΙΣ}
\def\period@term{ΠΕΡΙΟΔΟΥ}

\def\headmaster@term{Ο Διευθυντής}

\def\answerabbr@term{Απ.}

\def\exercise@term{Άσκηση}
\def\question@term{Ερώτηση}
\def\task@term{ΘΕΜΑ}
\def\solution@term{Λύση}
\def\hint@term{Υπόδειξη}

\endinput
%<package>}
%
%<package>\DeclareOption{grmath}{
%<package>  % %%%%%%%%%%%%%%%%%%%%%%%%%%%%%%%%%%%%%%%%%%%%%%%%%%
% schl package
% Mathematics operators in Greek.
% Adapted from grmath.sty
% by Apostolos Syropoulos.
% %%%%%%%%%%%%%%%%%%%%%%%%%%%%%%%%%%%%%%%%%%%%%%%%%%
\ProvidesFile{grmath.def}[2019/11/07 v0.1 Mathematics operators in Greek.]

\DeclareSymbolFont{groperators}{\encodingdefault}{\familydefault}{m}{n}
\SetSymbolFont{groperators}{bold}{\encodingdefault}{\familydefault}{bx}{n}

\def\sin{\mathop{\mathgroup\symgroperators ημ}\nolimits}
\def\cos{\mathop{\mathgroup\symgroperators συν}\nolimits}
\def\tan{\mathop{\mathgroup\symgroperators εφ}\nolimits}
\def\arcsin{\mathop{\mathgroup\symgroperators τοξημ}\nolimits}
\def\arccos{\mathop{\mathgroup\symgroperators τοξσυν}\nolimits}
\def\arctan{\mathop{\mathgroup\symgroperators τοξεφ}\nolimits}
\def\cot{\mathop{\mathgroup\symgroperators σφ}\nolimits}
\def\sec{\mathop{\mathgroup\symgroperators τεμ}\nolimits}
\def\csc{\mathop{\mathgroup\symgroperators στεμ}\nolimits}
\def\gcd{\mathop{\mathgroup\symgroperators ΜΚΔ}\nolimits}
\def\lcm{\mathop{\mathgroup\symgroperators ΕΚΠ}\nolimits}
\def\arccot{\mathop{\mathgroup\symgroperators τοξσφ}\nolimits}
\def\arcsec{\mathop{\mathgroup\symgroperators τοξτεμ}\nolimits}
\def\arccsc{\mathop{\mathgroup\symgroperators τοξστεμ}\nolimits}

%% %%%%%%%%%%%%%%%%%%%%%%%%%%%%%%
%% Write $x\to x_0$ under \lim
%% %%%%%%%%%%%%%%%%%%%%%%%%%%%%%%
\newcommand{\limdisplay}[1]{\displaystyle\lim_{#1}}
\endinput
%<package>}
%
%  \iffalse meta-comment
%%  Execute default options.
%  \fi
%<package>\ExecuteOptions{default}
%  
%<package>\ProcessOptions\relax
%%
% \iffalse meta-comment
%%
%%
%% %%%%%%%%%%%%%%%%%%%%%%%%%%%%%%%%%%%%%%%%%%%%%%%%%%%%%%%%%%%%%%%%%%%%%%%%%%%%%%%%%%%%%%%%%%
%% LOCALIZATION
%% To localize the package schl, define the following macros
%% with each term in your language. Put these definitions in
%% a file schl-<your-language>.def.
%% \grade@term             : Used by \examdetails and \examdetailsii.
%% \subject@term           : Used by \examdetails and \examdetailsii.
%% \testwriter@term        : Used by \examdetails.
%% \testsupervisor@term    : Used by \examdetails.
%% \date@term              : Used by \examdetails, \examdetailsii and \datefield.
%% \schoolyear@term        : Used by \examdetailsii.
%% \schoolyearabbr@term    : Used by \examdetails. An abbreviation for the word 'schoolyear'.
%% \examperiod@term        : Used by \examdetails.
%% \lastname@term          : Used by \examdetailsii.
%% \name@term              : Used by \examdetailsii.
%% \time@term              : Used by \examdetailsii.
%% \examnoabbr@term        : Used by \examdetailsii. An abbreviation for the exam number of a student.
%% \fullname@term          : Used by \fullname.
%% \deadline@term          : Used by \deadline.
%% \theoryheader@term      : Used by \theorypart. It is the word 'THEORY'.
%% \exerciseheader@term    : Used by \exercisepart. It is the word 'EXERCISES'.
%% \worksheet@term         : Used by \worksheettile.
%% \termtest@term          : Used by \examtitle. It is the word that corresponds to a
%%                           summary test during the year.
%% \exams@term             : Used by \finalexamheader. The word 'EXAMS'.
%% \period@term            : Used by \finalexamheader. The word 'PERIOD'.
%% \headmaster@term        : Used by \signatures.
%% \answerabbr@term        : Used by \answer. An abbreviation for the word 'answer'.
%% \exercise@term          : Used by the environment 'exercise'.
%% \question@term          : Used by the environment 'question'.
%% \task@term              : Used by the environment 'task'.
%% \solution@term          : Used by \solution.
%% \hint@term              : Used by \hint.
%% \points@term            : Used by \points. You might need to use an 'if' to distinguise
%%                           the cases for one or many points.
%% \trueabbr@term          : Used by \truefalselabel. The first letter of True.
%% \falseabbr@term         : Used by \truefalselabel. The first letter of False.
%%
%% Other internal macros that are used are 
%% \schl@wish             : A wish for good luck, usually printed in tests. Used by \wish.
%% \schl@headmaster       : The name of the headmaster. It is set with \headmaster.
%% \schl@teacher          : Teacher's name. It is set with \teacher.
%% \schl@subject          : The subject of the test, worksheet etc. (algebra, ...). It is set with \subject.
%%                          Used by \schoollogo, \examdetails and \examdetailsii
%% \schl@grade            : The grade. It is set with \grade. Used by \schoollogo, \examdetails 
%%                          and \examdetailsii
%% \schl@schldate         : The date that the produced document will be given to students.
%%                          Defined with \schldate and used by \examdetails and \examdetailsii.
%% \schl@examtime         : The time of the exam. It is set with \examtime and used by \examdetailsii.
%% \schl@schoolyear       : School year is set with \schoolyear and used by \examdetails and \examdetailsii.
%% \schl@authorityi, \schl@authorityii, \schl@authorityiii
%%                        : These three macros define a hierarhy of authorities with 
%%                          \schl@authorityi > \schl@authorityii > \schl@authorityiii.
%%                          e.x. \schl@authorityi might be the Ministry of Education, etc
%%                          They are set with \authorityi, \authorityii and \authorityiii
%%                          and are used in \authoritylogo.
%% \schl@school           : The name of the school.Is set with \school. It is used by \schoollogo
%%                          \authoritylogo.
%% %%%%%%%%%%%%%%%%%%%%%%%%%%%%%%%%%%%%%%%%%%%%%%%%%%%%%%%%%%%%%%%%%%%%%%%%%%%%%%%%%%%%%%%%%%
% \fi
%
% \iffalse meta-comment
%%
%%
%% Default line thickness for macro \blankspace.
% \fi
%<package>\def\schl@rulethickness{0.3pt}
% \iffalse meta-comment
%% Set \schl@rulethickness
% \fi
%<package>\def\rulethickness#1{%
%<package>  \def\schl@rulethickness{#1}
%<package> }
%
%   \iffalse meta-comment
%%
%% Macro \signatures is used to write the names of the teacher(s) and the headmaster. 
%% A handwritten signature is written in the right or above of those names.
%% \signatures uses a \parbox with length \signatureslength.
%% Each new name is written in a new line. The length between lines is
%% \signaturelineskip.
% \fi
%<package>\newlength{\signatureslength}
%<package>\setlength{\signatureslength}{13em}
%<package>\newlength{\signaturelineskip}
%<package>\setlength{\signaturelineskip}{7ex}
%
% \iffalse meta-comment
%%
%% Lengths \leftmatchwidth and \rightmatchwidth are used in macro \matchingque.
%% Both specify the length of a \parbox. \leftmatchwidth is the length of the left
%% column of the matching question, and \rightmatchwidth, is the length of the
%% right. The default length is 150pt.
% \fi
%<package>\newlength{\leftmatchwidth}
%<package>\newlength{\rightmatchwidth}
%<package>\setlength{\leftmatchwidth}{150pt}
%<package>\setlength{\rightmatchwidth}{150pt}
% 
% \iffalse meta-comment
%%
%% The following length is used inside the truefalse environment.
%% It specifies the length of a \parbox. The letters T (true) and F
%% are the text of this \parbox.
% \fi
%<package>\newlength{\truefalselength}
%<package>\setlength{\truefalselength}{50pt}
%
%
% \iffalse meta-comment
%%
%% Macro \truefalselabel prints T - F inside a \parbox.
%% \parbox's argument height is given the value 0pt.
%% This is done so that the top of the letters T - F allign
%% with the line of the true-false question.
%% This behaviour is heavily dependend on the font used and font size (\large here).
%%
%% \trueabbr@term and \falseabbr@term are the first letters of the words True, False.
% \fi
%<package>\def\truefalselabel{\parbox[t][0pt][c]{\truefalselength}{\large \trueabbr@term\hfill\falseabbr@term}}
%
% \iffalse meta-comment
%%
%% Define a new true-false toggle named \first. This is used inside the environment truefalse.
%% It is used to discriminate the first item from the rest.
% \fi
%<package>\newtoggle{first}
% 
% \iffalse meta-comment
%%
%% Macro \examdetailsii uses a 'tabular' environment. The space between rows is
% \fi
%<package>\renewcommand{\arraystretch}{1.5}
%
%
% \iffalse meta-comment
%%
%% \defaultletterspace is the default letter space percentage for macro \letterspace
% \fi
%<package>\def\defaultletterspace{10.0}
%<*driver>
\documentclass{ltxdoc}
\usepackage{fontspec}
\usepackage{xunicode}
\usepackage{xltxtra}
\usepackage{xgreek}
\setmainfont[Mapping=tex-text]{Linux Biolinum O}
\usepackage{schl}
\renewcommand{\abstractname}{Abstract}
\date{November, 2019}
\EnableCrossrefs
\CodelineIndex
\RecordChanges
\begin{document}
\DocInput{schl.dtx}
\end{document}
%</driver>
% \fi
%
%
% \changes{v0.1}{2019/06/22}{Initial version}
% \changes{v0.5}{2019/11/07}{2nd version}
% \changes{v1.0}{2019/11/07}{3rd version}
%
% \GetFileInfo{schl.sty}
%
% \DoNotIndex{\#,\$,\%,\&,\@,\\,\{,\},\^,\_,\~,\ }
% \DoNotIndex{\@ne}
% \DoNotIndex{\advance,\begingroup,\catcode,\closein}
% \DoNotIndex{\closeout,\day,\def,\edef,\else,\empty,\endgroup}
%
% \title{The \textsf{schl} package\thanks{This document
%   corresponds to \textsf{schl}~\fileversion,
%   dated~\filedate.}}
% \author{Tassos Tsesmetzis\\ \texttt{ttsesmetzis@gmail.com}}
%
% \maketitle
%
% \begin{abstract}
%   |schl| is a \XeLaTeX\  package that provides commands and environments suitable for
% document types that appear
%   in a classroom enviromnent. It's development is based on the Greek
%   educational practice, but it may be usefull in other contexts also.
% \end{abstract}
%
% \section{Introduction}
% Worksheets and tests are common document types in a classroom. |schl| package comes with
% macros that facilitate the creation of these documents. It has list environments for
% questions, exercises and tasks. Other environments of the package can be used for tickable or
% multiple choice answers. There are also commands for typesetting solutions, hints
% and answers to exercises.
%
% Furthermore, you can set the name of the teacher,  subject, grade, headmaster, school, date, school year
% and use these to print school's logo or information about an exam. |schl| has commands
% to typeset headers for each document type, a macro for typing the points of an exercise and two
% commands for blank space. There is also a macro for typesetting a wish for good luck!
%
% |schl| is based on the Greek school practice. It redefines in Greek the common math macros
% |\sin, \cos, \tan, \cot| and |\gcd|. Also, it provides the math operator |\lcm| for the least
% common multiple of integers. A characteristic of Greek school mathematics, is that |\lim|
% operator appears in display mode. |schl| offers a macro for this.
%
% By default, |schl| prints all macros that accept text as |undefined|. As of this version (\fileversion), Greek
% is the only supported language. You can set it with the option |greek|. Other languages can be supported
% by redefining package's internal macros.
%
% |schl| loads the packages |fontspec, enumitem, mdframed| and |amsmath|. It is written for \XeLaTeX, but
% can be used by any system that supports |fontspec|.
% \section{Macros}
% \subsection{Mathematics}
% The option |grmath| provides common mathematics operators in Greek. 
% Specifically redefines in Greek the trigonometric operators
% |\sin|, |\cos|, |\tan|, |\cot|, |\arcsin|, |\arccos|, |\arctan|, |\cot|, |\sec|, |\csc|, |\arccot|, |\arcsec| and |\arccsc|.
% Also, provides the arithmetic operators |\gcd| and |\lcm| for \textsf{greatest common divisor} and
% \textsf{least common multiple} .
%
% \DescribeMacro{\limdisplay} Command |\limdisplay| \marg{text} prints \meta{text} under |\lim|.
% 
% \subsection{Blank space}
% \DescribeMacro{\lowerdots}\DescribeMacro{\blankspace} Usually, we  need to designate blank space in a document. |schl| package has two commands for this. The first one |\lowerdots| \oarg{length}\marg{number}, prints \meta{number} dots. Optional argument \meta{length} sets the deviation from base line. It's  default value is |-0.3ex|.
% \iffalse meta-comment
%% This command prints dots that represent a blank space.
%% The first argument accepts a length. The deviation from the baseline.
%% Second argument is the number of dots that are printed.
%% From egreg's answer on tex.stackexchange.org
%% https://tex.stackexchange.com/questions/300207/repeat-characters-n-times
%% accessed June 06, 2019
% \fi
%    \begin{macrocode}
 \newcommand\lowerdots[2][-0.3ex]{%
   \begingroup
   \lccode`m=`.\relax
   \raisebox{#1}{\lowercase\expandafter{\romannumeral\number\number#2 000}}%
   \endgroup
 }
%    \end{macrocode}
%
% |\blankspace| \oarg{length}\marg{linelength} prints a line with length \meta{linelength}. The optional argument is the deviation from the base line and it's default value is |-0.3ex|. |\schl@rulethickness| is the default thickness for all |\blankspace| lines.
% \iffalse meta-comment
%%
%% A line that represents blank space.
%% First argument: deviation from base line.
%% Second argument: line width.
% \fi
%    \begin{macrocode}
\newcommand\blankspace[2][-0.3ex]{%
  \raisebox{#1}{\rule{#2}{\schl@rulethickness}}
 }
%    \end{macrocode}
%
% \subsection{Lists}
% |schl| package defines seven types of lists. These are |question|, |exercise|, |schltask|, |multichoice|, |tickchoice|, |truefalse| and |matchique|. |tickchoice|  comes also with a stared version |tickchoice*|. All of them depend on the package |enumitem|.
%
% \DescribeEnv{question}\DescribeEnv{exercise}\DescribeEnv{schltask}  These environments are  |enumerate|-like lists. List's |\item| is of the form \meta{type} \meta{counter}, where |type| is |\question@term| for |question|, |\exercise@term| for |exercise| and |\task@term| for |schltask|. \meta{counter} is the internal counter of the environment. 
% \iffalse meta-comment
%%
%% A list for questions
% \fi
%    \begin{macrocode}
\newlist{question}{enumerate}{1}
\setlist*[question]{%
  align=left,
  label=\normalsize\bf \question@term\  \arabic*.,
  wide,
  leftmargin=0pt,
  labelindent=0pt
}
%    \end{macrocode}
%
% \iffalse meta-comment
%%
%% A list for exercises.
% \fi
%    \begin{macrocode}
\newlist{exercise}{enumerate}{1}
\setlist*[exercise]{%
  align=left,
  label=\normalsize\bf\exercise@term\  \arabic*.,
  wide,
  leftmargin=0pt,
  labelindent=0pt
}
%    \end{macrocode}
%
% \iffalse meta-comment
%%
%% A list for tasks.
% \fi
%    \begin{macrocode}
\newlist{schltask}{enumerate}{1}
\setlist*[schltask]{%
  align=left,
  label=\normalsize\bf\letterspace{\defaultletterspace}\task@term\  \Alph*,
  wide,
  leftmargin=0pt,
  labelindent=0pt
}
%    \end{macrocode}
%
% \DescribeMacro{\letterspace} The macro |\letterspace|\marg{number} is used to set the horizontal space of adjacent characters in a word. It is based on the |\addfontfeature| macro from the package |fontspec|.
% The argument \meta{number} is a percentage of the font size. In |schl| package is used to set the
% space between capital word letters.
%  \iffalse meta-comment
%%
%%  Space of capital word letters
%% \addfontfeature is from fontspec package.
%  \fi
%    \begin{macrocode}
\def\letterspace#1{\addfontfeature{LetterSpace=#1}}
%    \end{macrocode}
%
% \DescribeEnv{multichoice} The |multichoice| environment is used to typeset multiple choice answers.
% \iffalse meta-comment
%%
%% Insert multiple choice anwers.
%% Leave a blank line before and after \begin{multichoice} ... \end{multichoice}
%% for proper rendering.
% \fi
%    \begin{macrocode}
\newlist{multichoice}{enumerate*}{1}
\setlist*[multichoice]{
  labelindent=\parindent,
  label=\Alph*.,
  itemjoin=\hspace{\fill},
  before=\hspace{\fill},
  after=\hspace{\fill}
}
%    \end{macrocode}
%
% \DescribeEnv{tickchoice}\DescribeEnv{tickchoice*} The environments |tickchoice| and |tickchoice*| are variants of the |itemize| list. For both cases, each item is preceded by a square. |tickchoice| stacks items vertically,
% \iffalse meta-comment
%%
%% Leave a blank line before and after \begin{tickchoice} ... \end{tickchoice}
%% for proper rendering
% \fi
%    \begin{macrocode}
\newlist{tickchoice}{itemize}{1}
\setlist[tickchoice]{labelindent=\parindent,label={\large$\square$}}
%    \end{macrocode}
% while |tickchoice*| stacks them horizontally.
%    \begin{macrocode}
\newlist{tickchoice*}{itemize*}{1}
\setlist*[tickchoice*]{
  labelindent=\parindent,
  label={\large$\square$},
  itemjoin=\hspace{\fill},
  before=\hspace{\fill},
  after=\hspace{\fill}
}
%    \end{macrocode}
% \DescribeEnv{truefalse} |truefalse| is a variant of the |enumerate| environment. Each |\item| is divided in two parts. The
% first part is the text that follows the |\item| macro. The second part is a |\parbox| that prints |\trueabbr@term| and
% |\falseabbr@term|.
% \iffalse meta-comment
%% An environment to typeset true-false questions. Is is defined with the enumitem package.
%% Each question is followed by the letters \trueabbr@term and \falseabbr@term.
%% It uses the calc package
%% 
%% Adapted from cfr's answer on
%% https://tex.stackexchange.com/questions/164613/adding-code-at-the-end-of-each-list-item
%% accessed 11/07/2019
% \fi
%    \begin{macrocode}
\newlist{truefalse}{enumerate}{1}
\setlist[truefalse]{label={\bf \arabic*.},%
  before*={%
    \let\defaultitem\item%      Save the standard definition of \item in a macro.
    \toggletrue{first}%                 Set the first toggle with initial value true.
    \def\item{%                 
      \iftoggle{first}{%
        \togglefalse{first}%              Set the first toggle to take the value false.
        \defaultitem\begin{minipage}[t]{0.8\linewidth minus \truefalselength}%
        }{%
        \end{minipage}\hfill\truefalselabel\defaultitem%
        \begin{minipage}[t]{0.8\linewidth minus \truefalselength}%
        }
      }% new, temporary defition of \item
    },
    after*={%   This takes care of adding the fill for the final item on
      %         the list and just makes sure that \item is reset to its standard definition
    \end{minipage}\hfill\truefalselabel% fill for final item in list
    \let\item\defaultitem% restore standard definition of \item
  }%
}
%    \end{macrocode}
%
% \DescribeMacro{matchingque} The macro |\matchingque|\marg{CSV}\marg{CSV} is used to typeset matching questions. \meta{CSV}
% are comma separated values. The \meta{CSV}s of the first argument are the parts of the matching questions that will 
% be print in the left column. Similarly, the \meta{CSV} of the second argument are going to be printed on the right
% column of the matching questions.
%    \begin{macrocode}
\newcommand\matchingque[3][300pt]{%
  \begin{center}
    \parbox[c]{#1}{
      \parbox[c]{\leftmatchwidth}{%
        \begin{leftmatching}
          \@for\tmp:=#2%
          \do{%
          \item \tmp
          }
        \end{leftmatching}
      }\hfill%
      \parbox[c]{\rightmatchwidth}{%
        \begin{rightmatching}
          \@for\tmp:=#3%
          \do{%
          \item \tmp
          }
        \end{rightmatching}
      }
    }
  \end{center}
}
%    \end{macrocode}
%
% \DescribeEnv{leftmatching}\DescribeMacro{rightmatching} Environments |leftmatching| and |rightmatching| are used to
% typeset each column in |\matchingque|.
%    \begin{macrocode}
\newlist{leftmatching}{enumerate}{1}
\newlist{rightmatching}{enumerate}{1}
\setlist*[leftmatching]{label=\bf\Alph*.}
\setlist*[rightmatching]{label=\bf\arabic*.}
%    \end{macrocode}
% \subsection{Answers, solutions and hints}
% \DescribeMacro{\answer} Macro |\answer|\marg{text} prints |(\answerabbr@term \meta{text})| at the right end of the current line.
% \iffalse meta-comment
%%
%% Write the answer of an exercise.
% \fi
%    \begin{macrocode}
\newcommand\answer[1]{%
  \hfill{\footnotesize (\answerabbr@term: #1)}
}
%    \end{macrocode}
%
% \DescribeMacro{\solution} Macro |\solution|\marg{text} is used to typeset the solution of an exercise. 
% \iffalse meta-comment
%%
%% With this command you write the solution of an exercise.
% \fi
%    \begin{macrocode}
\newcommand\solution[1]{%
  \par\noindent\phantom{.}\hfill\textbf{\solution@term}\hfill\phantom{.}\par%
  \noindent #1
}
%    \end{macrocode}
%
% \DescribeMacro{\hint} |schl| provides the macro |\hint|\marg{text} for typesetting exercise hints.
% \iffalse meta-comment
%%
%% A macro for exercise hints.
% \fi
%    \begin{macrocode}
\newcommand\hint[1]{%
  \par{\scriptsize\noindent\textbf{\hint@term:} #1}%
}
%    \end{macrocode}
%
% \DescribeMacro{\deadline} A feature of homework assignments is a deadline date. |\deadline|\marg{date} prints |\deadline@term| followed by argument \meta{date}.
% \iffalse meta-comment
%%
%% Print the deadline of an assignment
% \fi
%    \begin{macrocode}
\newcommand\deadline[1]{%
  \noindent{{\bf\normalsize\deadline@term}: #1}
}
%    \end{macrocode}
%
% \subsection{Titles and headers}
% \DescribeMacro{\heading} Common document types  in a school environment are the worksheet, various tests and final written exams. The macro |\heading|\marg{text} gives a generic header for all these documents.
% \iffalse meta-comment
%%
%% Large centered text in bold face.
% \fi
%    \begin{macrocode}
\newcommand\heading[1]{%
  \begin{center}
    {\bf\large #1}
  \end{center}
}
%    \end{macrocode}
%
% \DescribeMacro{\worksheettitle} Macro |\worksheettitle|\marg{text} sets the title of a worksheet. It appends \meta{text} to |\worksheet@term|.
% \iffalse meta-comment
%%
%% The title of a worksheet.
%% Argument is a text to be appended in \worksheet@term
% \fi
%    \begin{macrocode}
\newcommand\worksheettitle[1]{%
  \heading{\worksheet@term\  #1}
}
%    \end{macrocode}
%
% \DescribeMacro{\examtitle} |\examtitle|\oarg{text}\marg{text} is used to set the title of tests. The optional argument has the default value |\termtest@term|. 
% \iffalse
%%
%% Exam and term test title.
% \fi
%    \begin{macrocode}
\newcommand\examtitle[2][\termtest@term]{%
  \heading{#1 #2}
}
%    \end{macrocode}
%
% \DescribeMacro{\finalexamheader} Titles for end year exams have a standardized form in Greek schools. |\exams@term| is followed by information about the exam. Then comes |\period@term| with the exam period after it. |\finalexamheader|\marg{info}\marg{period} is used for these cases.
% \iffalse meta-comment
%%
%% Default headers for final exams.
% \fi
%    \begin{macrocode}
\newcommand\finalexamheader[2]{%
  \heading{\letterspace{\defaultletterspace} #1 \exams@term\\[0.5ex] \period@term\  #2}
}
%    \end{macrocode}
%
% \DescribeMacro{\schl@framedbox} |\schl@framedbox|\marg{text} prints \meta{text} in a centered frame box. It is
% used by |\theorypart| and |\exercisepart|.
% \iffalse meta-comment
%%
%% A generic header for centered framed text.
% \fi
%    \begin{macrocode}
\newcommand\schl@framedbox[1]{%
  \begin{center}
    \fbox{\large{\bf\letterspace{\defaultletterspace} #1} }%
  \end{center}
}
%    \end{macrocode}
%
% \DescribeMacro{\theorypart}\DescribeMacro{\exercisepart} Sometimes theory and exercise sections constitute
% a written test. Macros |\theorypart| and |\exercisepart| print headers for those parts.
%    \begin{macrocode}
\newcommand\theorypart{%
  \schl@framedbox{\theoryheader@term\!}
}
%    \end{macrocode}
% and
%    \begin{macrocode}
\newcommand\exercisepart{%
  \schl@framedbox{\exerciseheader@term\!}
}
%    \end{macrocode}
% 
% \subsection{School information}
% \DescribeMacro{\school}\DescribeMacro{\headmaster}\DescribeMacro{\teacher}\DescribeMacro{\subject}\DescribeMacro{\grade}\DescribeMacro{\schoolyear}\DescribeMacro{\schldate}\DescribeMacro{\examtime} The macros |\school|\marg{text}, |\headmaster|\marg{name}, |\teacher|\marg{name}, |\subject|\marg{text}, |\grade|\marg{text}, |\schoolyear|\marg{year}, |\schldate|\marg{date} and |\examtime|\marg{time} define and set the value of internal macros.
% \iffalse meta-comment
%%
%% Set subject, grade, date, school year etc.
% \fi
%    \begin{macrocode}
\newcommand\school[1]{\def\schl@school{#1}}
\newcommand\headmaster[1]{\def\schl@headmaster{#1}}
\newcommand\teacher[1]{\def\schl@teacher{#1}}
\newcommand\subject[1]{\def\schl@subject{#1}}
\newcommand\grade[1]{\def\schl@grade{#1}}
\newcommand\schoolyear[1]{\def\schl@schoolyear{#1}}
\newcommand\schldate[1]{\def\schl@schldate{#1}}
\newcommand\examtime[1]{\def\schl@examtime{#1}}
%    \end{macrocode}
%
% \DescribeMacro{\authorityi}\DescribeMacro{\authorityii}\DescribeMacro{\authorityiii} In a similar vein, |\authorityi|\marg{text}, |\authorityii|\marg{text} and |\authorityiii|\marg{text} define the internal macros |\schl@authorityi|, |\schl@authorityii| and |\schl@authorityiii|. 
%    \begin{macrocode}
\newcommand\authorityi[1]{\def\schl@authorityi{#1}}
\newcommand\authorityii[1]{\def\schl@authorityii{#1}}
\newcommand\authorityiii[1]{\def\schl@authorityiii{#1}}
%    \end{macrocode}
%
% \subsection{Other macros for tests}
% \DescribeMacro{\points} |\points|\oarg{macro}\marg{number} is used to designate the points of an exercise. \marg{number} is the number of points for the current exercise, while \oarg{macro} can be used to control the space just before the points.
% \iffalse meta-comment
%%
%% Print the points of an exescise.
% \fi
%    \begin{macrocode}
\newcommand{\points}[2][\hfill]{%
#1(\textbf{\footnotesize \points@term{#2}\  #2})
}
%    \end{macrocode}
%
% \DescribeMacro{\fullname} |\fullname|\marg{text} prints |\fullname@term| followed by \meta{text}. 
% \iffalse meta-comment
%%
%% A field for writing the full name.
% \fi
%    \begin{macrocode}
\newcommand\fullname[1]{%
  \noindent{\normalsize\fullname@term :} #1
}
%    \end{macrocode}
%
% \DescribeMacro{\datefield} Similarly, |\datefield|\marg{text} prints |\date@term| with \meta{text} after it.
% \iffalse meta-comment
%%
%% A field for writing the date.
% \fi
%    \begin{macrocode}
\newcommand\datefield[1][0]{%
  \noindent{\normalsize\date@term :}
}
%    \end{macrocode}
%
% \DescribeMacro{\schoollogo} |\schoollogo|\marg{width} prints |\schl@school, \schl@grade, \schl@subject| and |\schl@teacher|. \meta{width} is the length of the |\parbox|.
% \iffalse meta-comment
%%
%% School logo
%% Argument 1 is the length of the parbox.
% \fi
%    \begin{macrocode}
\def\schoollogo#1{%
  \parbox[t]{#1}{%
    \schl@school\\%
    \schl@grade\\%
    \schl@subject\\%
    \schl@teacher
  }
}
%    \end{macrocode}
%
% \DescribeMacro{\authoritylogo} |\authoritylogo|\oarg{number} prints |\sch@authorityi, \sch@authorityii, \sch@authorityiii| and |\schl@school|. Argument \meta{number} is a multiplier for |\baselineskip|. This spaces is added above macro.
% \iffalse meta-comment
%%
%% Logo for exams.
%% Argument is a multiplier for \baselineskip.
% \fi
%    \begin{macrocode}
\newcommand\authoritylogo[1][1.5]{%
  \noindent\parbox[t][\height]{0.4\textwidth}{%
    \centering%

    \vspace{#1\baselineskip}

    {\schl@authorityi}

    \vspace{3\lineskip}
    
    {\footnotesize\schl@authorityii}

    \vspace{2\lineskip}
    
    {\footnotesize\schl@authorityiii}

    \vspace{3\lineskip}

    {\small\letterspace{\defaultletterspace}\MakeUppercase{\schl@school}}
  }
}
%    \end{macrocode}
% \DescribeMacro{\examdetails}\DescribeMacro{\examdetailsii} Written exam documents contain information about the period of the exam,  subject, grade, writer of the test, supervisors of the exam and date. |schl| package has the macros |\examdetails|\marg{text} and |\examdetailsii| for printing this information. Argument \meta{text} of |\examdetails| is the exam period.
% \iffalse meta-comment
%%
%% Print details of the exam.
%% First argument is the line width for the frame.
%% Second argument is the length of the \parbox.
% \fi
%    \begin{macrocode}
\newcommand\examdetails[2][3pt]{%
  \parbox[t]{#2}{
    \begin{mdframed}[linewidth=#1]
      \normalsize%
      {%
        \bf\letterspace{\defaultletterspace}%
        \schoolyearabbr@term:\hspace{3pt}\schl@schoolyear
      }\\[1.0ex]
      \textbf{\grade@term:}\hspace{3pt}\schl@grade\\[1.0ex]
        \textbf{\subject@term:}\hspace{3pt}\schl@subject \\[1.0ex]
        \textbf{\testwriter@term:}\hspace{3pt}\schl@teacher\\[1.0ex]
        \textbf{\testsupervisor@term:}\\[1.0ex]
        \textbf{\date@term:}\hspace{3pt}\schl@schldate
    \end{mdframed}
  }
}
%    \end{macrocode}
% and
%    \begin{macrocode}
\newcommand\examdetailsii{%
  \parbox[t]{0.53\linewidth}{%
    \begin{center}%
      \underline{\bf\letterspace{\defaultletterspace}\schoolyear@term\  \schl@schoolyear}%
    \end{center}
    \begin{tabular}{|r|c|r|c|}
      \hline
      {\bf\letterspace{\defaultletterspace}\lastname@term:} & \multicolumn{3}{|c|}{} \\
      \hline
      {\bf\letterspace{\defaultletterspace}\name@term:} & \multicolumn{3}{|c|}{} \\
      \hline
      {\bf\letterspace{\defaultletterspace}\examnoabbr@term:} & %
                      & {\bf\letterspace{\defaultletterspace}\MakeUppercase{\grade@term}:}
                      & \schl@grade \\
      \hline
      {\bf\letterspace{\defaultletterspace}\MakeUppercase{\subject@term}:} %
                      & \multicolumn{3}{|c|}{\schl@subject} \\
      \hline
      {\bf\letterspace{\defaultletterspace}\MakeUppercase{\date@term}:} & \schl@schldate
                      & {\bf\letterspace{\defaultletterspace}\time@term:} %
                      & \schl@examtime\\
      \hline
    \end{tabular}
  }
}
%    \end{macrocode}
%
% \DescribeMacro{\signatures} Some types of written tests end with the names of the headmaster and the teacher(s) followed by handwritten signatures. |\signatures|\oarg{role}\marg{signer(s)} prints the name(s) of the \meta{signer(s)} under a line with the \meta{role} of the signer(s). |\signatureslength| is the length of the |\signatures| block.
%    \begin{macrocode}
\newcommand\signatures[2][\headmaster@term]{%
  \parbox[t]{\signatureslength}{%
    \setlength \baselineskip{\signaturelineskip}
    \begin{center}
      #1 \\ #2
    \end{center}
  }
}
%    \end{macrocode}
%
% \DescribeMacro{\wish} |\wish| prints |\schl@wish|, a default wish for tests.
% \iffalse meta-comment
%%
%% A wish
% \fi
%    \begin{macrocode}
\newcommand\wish[1][\schl@wish]{%
  \begin{center}
      {\LARGE\bf #1}
  \end{center}
}
%    \end{macrocode}
%
% \PrintChanges
% \StopEventually{\PrintIndex}
% \Finale
\endinput