\documentclass[12pt,a4page]{article}
\usepackage{xunicode}
\usepackage{xltxtra}
\usepackage{xgreek}

\setmainfont[Mapping=tex-text]{Linux Biolinum O}

\usepackage[width=0.9\paperwidth,height=0.9\paperheight,centering]{geometry}
\usepackage{color}
\usepackage{amsfonts}
\usepackage{amsmath}
\usepackage{graphicx}
\usepackage{bm}
\usepackage{enumerate}
\pagestyle{empty}

% Load the package 'schl'.
\usepackage[greek]{schl}

\begin{document}

\heading{Tests for the \textit{schl} package}

\begin{enumerate}
\item Blank space\\
  I am  \lowerdots{3} years old. Banach is a mathematician from \lowerdots{20}. Change the vertical position \lowerdots[0.5ex]{10}!\\ Also in mathematical expressions $\cos\frac\pi4 = \lowerdots{4}$.\\
  I am  \blankspace{2ex} years old. Banach is a mathematician from \blankspace{15ex}. Change the vertical position \blankspace[0.5ex]{5ex}!\\ Also in mathematical expressions $\cos\frac\pi4 = \blankspace{4ex}$.\\
\item Environment \texttt{exercise} is for typesetting exercises.
  \begin{exercise}
  \item  Να γράψετε όλους τους πρώτους αριθμούς που είναι μικρότεροι του $100$.
  \item Έχουμε κόκκινες και πράσινες χάντρες για να φτιάξουμε ένα βραχιόλι. Αν χρησιμοποιήσουμε όλες τις κόκκινες και βάλουμε τον ίδιο αριθμό από πράσινες, τότε, θα περισσέψει μια πράσινη χάντρα. Αν χρησιμοποιήσουμε όλες τις πράσινες και βάλουμε μισές κόκκινες από τον αριθμό των πράσινων, τότε, θα περισσέψουν δύο κόκκινες χάντρες. Πόσες κόκκινες και πόσες πράσινες χάντρες έχουμε;
  \item Δείξτε ότι το άθροισμα των γωνιών ενός τριγώνου είναι $180^\circ$.
  \end{exercise}
\item Environment \texttt{schltask} can be used for summative tests. 
  \begin{schltask}
  \item Λύστε την εξίσωση $x^2 - 3x + 2 = 0$.
  \item Αποδείξτε το Πυθαγόρειο θεώρημα.
  \item Δείξτε ότι οι διάμεσοι ενός τριγώνου συντρέχουν.
  \end{schltask}
\item The macro \verb/\answer/ is used to typeset the answer of an exercise.
  \begin{exercise}
  \item Υπολογίστε το άθροισμα 1 + 1.\\\phantom{.}\answer{2}
  \end{exercise}
\item With the macro \texttt{solution}, we write the solution of an exercise.
  \begin{exercise}
  \item Δείξτε ότι το πλήθος των πρώτων είναι άπειρο.
    \solution{%
      Υποθέτουμε ότι υπάρχει πεπερασμένο πλήθος πρώτων αριθμών $p_1,\ldots,p_\nu$. Θεωρούμε τον ακέραιο \ldots}
  \end{exercise}
  \begin{schltask}
  \item Να δείξετε ότι η ευθεία $y = x$ διέρχεται από την αρχή των αξόνων.
    \solution{%
      Αρκεί να δείξουμε ότι οι συντεταγμένες του σημείου $O(0\,,\,0)$ επαληθεύουν την εξίσωση της ευθείας.}
  \end{schltask}
\item Set points to exercises:
  \begin{schltask}
  \item \points{25}\\
    Διατυπώστε και αποδείξτε το θεώρημα του Bolzano.
  \item  Θεωρούμε τη συνάρτηση $f:\mathbb{R}\rightarrow\mathbb{R}$ με τύπο $f(x) = \frac1{x-1}$.
    \begin{enumerate}
    \item Βρείτε το πεδίο ορισμού της $f$.\points{10}
    \item Υπολoγίστε το $f(3)$.\points{1}
    \end{enumerate}
  \end{schltask}
  \item Environment \texttt{question}:.
    \begin{question}
    \item Υπάρχει μικρότερος πραγματικός αριθμός;
    \item Υπάρχει μεγαλύτερος αρνητικός αριθμός;
    \end{question}
\item Hints:
  \begin{exercise}
  \item Δείξτε ότι ανάμεσα σε οποιουσδήποτε δύο ρητούς αριθμούς, υπάρχει ρητός.
  \hint{%
    Θεωρούμε τους ρητούς $\rho_1 < \rho_2$. Ορίζουμε τον πραγματικό $\frac{\rho_1 + \rho_2}2$. Στη συνέχεια δείχνουμε ότι ο $x$ είναι \ldots}
\item Δείξτε ότι $(\alpha + \beta)^2 = \alpha^2 + 2 \alpha \beta + \beta^2$.
  \hint{%
    Έχουμε $(\alpha + \beta)^2 = (\alpha + \beta) \cdot (\alpha + \beta) = \ldots$}
  \end{exercise}
\item Environment \texttt{multichoice} is for multiple choice questions:
  % leave a blank line
  
  \begin{multichoice}
  \item choice 1
  \item choice 2
  \end{multichoice}
  % leave a blank line

  Another example
  % leave a blank line

  \begin{multichoice}
  \item choice 1
  \item choice 2
  \item choice 3
  \end{multichoice}

  Or

  \begin{multichoice}[before=\hspace{3em},itemjoin=\hspace{3em},label=\bf\arabic*{})]
  \item  this is a very long choice 1
  \item this is an even longer choice 2\\\hspace*{9em}
  \item this is a remarkably long choice 3
  \end{multichoice}
\item Environment \texttt{tickchoice}. Horizontal

  \begin{tickchoice*}
  \item choice A
  \item choice B
  \item choice C
  \end{tickchoice*}

  and vertical
  
  \begin{tickchoice}
  \item choice A
  \item choice B
  \item choice C
  \end{tickchoice}
\item A wish for good luck
  \wish
  Setting the text. Macro \verb/\letterspace/ sets the space between adjucent letters
  \makeatletter
  \def\schl@wish{\letterspace{10} ΚΑΛΗ ΤΥΧΗ}
  \makeatother
  \wish
\item Write the name and date:\\
  \fullname{}\\
  \datefield{} \\[1ex]
  Also, with dots or a line for blank space:\\
  \fullname{\lowerdots{40}}\\ \datefield{\blankspace{10em}}\\[1ex]
  We could use \\
  \setdate{28 Μαΐου 2020}
  \datefield{\getdate}
\item Exercise deadline: \hspace{3em} \deadline{2/2/2058}\\
\item Set the duration of a test:\\ \duration{10'}
\item Add a remark in a document:\\ \remark{Αυτή είναι μια παρατήρηση.}
\item Add a reminder in a document:\\ \reminder{Εδώ ξεκινά μια υπενθύμιση.}
\item Header for the theory part of a test: \theorypart
  Header for the exercise part of a test: \exercisepart
\item Set the title of a worksheet
  \worksheethd{}
  or
  \worksheethd{στην παράγραφo \S A.2.3}
\item Teacher/headmaster signatures:\\
  \signatures{Georg Cantor}
  \hfill
  \signatures[Οι Εισηγητές]{Αλφαβήτας Γαμαδέλτας,Εψιλονζήτας Ηταθήτας}
\item Headers for tests:
  \examhd{}
  \examhd{Α' τετραμήνου}
  \examhd[Τεστ]{στο κεφάλαιο 1}
\item Header for end year summative tests:
  \finalexamhd{ΓΡΑΠΤΕΣ ΕΠΑΝΑΛΗΠΤΙΚΕΣ}{ΜΑΪΟΥ -- ΙΟΥΝΙΟΥ}
\item Logo of the  exams

  \authoritylogo\\
  
  or if we set \verb+\authorities+ and \verb+schl@authorities+:

  \school{%
    ΕΛΛΗΝΙΚΗ ΔΗΜΟΚΡΑΤΙΑ

    \vspace{3\lineskip}

    ΥΠΟΥΡΓΕΙΟ ΠΑΙΔΕΙΑΣ, ΕΡΕΥΝΑΣ \& ΘΡΗΣΚΕΥΜΑΤΩΝ

    \vspace{3\lineskip}

    ΔΙΕYΘΥΝΣΗ Β/ΘΜΙΑΣ ΕΚΠΑΙΔΕΥΣΗΣ ΜΕΣΣΗΝΙΑΣ
  }
  \school{Λύκειο Καλαμάτας}
  \authoritylogo
\item School logo

  \school{ΓΥΜΝΑΣΙΟ ΠΑΤΡΩΝ}
  \grade{Β' Γυμνασίου}
  \subject{Μαθηματικά}
  \teacher{Ήρων από την Αλεξάνδρεια}
  \schoollogo{200pt}
\item True-false type questions with the environment \verb|truefalse|
  \begin{truefalse}
  \item kjahs naoisjh nmaksjnd njaksjn dnamksdoh n ash nda
    ias doasj d jjsn ndijewh nasusfd has hujh djnjdi haiusd i
  \item kjahs naoisjh nmaksjnd njaksjn dnamksdoh n ash nda
    ias doasj d jjsn ndijewh nasusfd has hujh djnjdi haiusd i
    kjahs naoisjh nmaksjnd njaksjn dnamksdoh n ash nda
    ias doasj d jjsn ndijewh nasusfd has hujh djnjdi haiusd i
    kjahs naoisjh nmaksjnd njaksjn dnamksdoh n ash nda
  \item  ias doasj d jjsn ndijewh nasusfd has hujh djnjdi haiusd i
  \item     ias doasj d jjsn ndijewh nasusfd has hujh djnjdi haiusd i
    kjahs naoisjh nmaksjnd njaksjn dnamksdoh n ash nda
    ias doasj d jjsn ndijewh nasusfd has hujh djnjdi haiusd i
    kjahs naoisjh nmaksjnd njaksjn dnamksdoh n ash nda
  \end{truefalse}
\item Matching questions:
  \setlist*[leftmatching]{label=}
  \setlist*[rightmatching]{label=}
  \setlength{\rightmatchwidth}{200pt}
  \matchingque[320pt]{παιδί,χταπόδι,παιχνίδι}{θάλασσα,κατάστημα,διάστημα,διάβασμα,ψωμί, σαλάμι}
\end{enumerate}
\end{document}