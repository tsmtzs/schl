\documentclass[10pt,a4page]{article}
\usepackage{xunicode}
\usepackage{xltxtra}
\usepackage{xgreek}

\setmainfont[Mapping=tex-text]{Linux Biolinum O}

\usepackage[width=0.9\paperwidth,height=0.9\paperheight,centering]{geometry}
\usepackage{xcolor}
\usepackage{amsfonts}
\usepackage{amsmath}
\usepackage{graphicx}
\usepackage{bm}
\usepackage{enumerate}
\usepackage{listings}

\lstset{
  basicstyle=\footnotesize\ttfamily,
  columns=flexible,
  breaklines=true,
  backgroundcolor=\color{blue!10},
  extendedchars=true,
  numbers=none,
  language=TeX,
  belowskip=-1\medskipamount
}

\pagestyle{empty}

% Load the package 'schl'.
\usepackage[english]{schl}

% Change the space between adjacent rows in a tabular environment.
\renewcommand{\arraystretch}{2.5}

\author{Tassos Tsesmetzis}
\title{An overview of the \texttt{schl} package}
\date{}

% A macro for table column headings
\newcommand{\columnhd}[1]{%
  \hfill\textit{\large #1}\hfill\phantom{.}
}
\def\cellwidth{0.4\textwidth}
% Insert code and text examples in tables:
\newcommand{\miniexample}[2][c]{%
  \parbox[#1]{\cellwidth}{#2}
}
\newcommand{\codeexample}[2][c]{%
  \miniexample[#1]{\lstinputlisting{#2}}
}
\newcommand{\textexample}[2][c]{%
  \small
  \colorbox[gray]{0.9}{\miniexample[#1]{\input{#2}}}
}

\newcommand{\examplerow}[2][10pt]{%
  \par\noindent\phantom{}\hfill\codeexample{#2}\hspace{#1}\textexample{#2}\hfill\phantom{}
}

\begin{document}
\maketitle

\verb|schl| is a \XeLaTeX\, package that provides commands and environments suitable for document types that appear in a classroom enviromnent. It's development is based on the Greek school educational practice, but it may be usefull in other contexts also. This document offers a quick view of working examples for \verb|schl|'s marcos. If we load the package passing the parameter \verb|greek|, several macros will be printed in Greek. These are
defined in \verb|languages/sch-greek.def|. If you want to set them in a different language modify the \verb|languages/sch-template.def| file.

\begin{enumerate}
\item Blank space is designated with the macros \verb|\lowerdots| and \verb|\blankspace|.
  \examplerow{exampleBlankspace01.tex}
  \examplerow{exampleBlankspace02.tex}
  \examplerow{exampleBlankspace03.tex}
  \examplerow{exampleBlankspace04.tex}

  % %%%%%%%%%%%%%%%%%%%%%%%%%%%%%%%%%%%%%%%%%%%%%%%%%% 
  % Environment 'exercise'
  % %%%%%%%%%%%%%%%%%%%%%%%%%%%%%%%%%%%%%%%%%%%%%%%%%% 
\item With the environment \texttt{exercise} you can typeset exercises.
  \examplerow{exampleEnvirExercise.tex}
  % %%%%%%%%%%%%%%%%%%%%%%%%%%%%%%%%%%%%%%%%%%%%%%%%%% 
  % Environment schltask'
  % %%%%%%%%%%%%%%%%%%%%%%%%%%%%%%%%%%%%%%%%%%%%%%%%%% 
\item The environment \texttt{schltask} can be used for summative tests.
  \examplerow{exampleEnvirSchltask.tex}
  % %%%%%%%%%%%%%%%%%%%%%%%%%%%%%%%%%%%%%%%%%%%%%%%%%% 
  % Macro \asnswer
  % %%%%%%%%%%%%%%%%%%%%%%%%%%%%%%%%%%%%%%%%%%%%%%%%%% 
\item The macro \verb/\answer/ is used to typeset the answer of an exercise.
  \examplerow{exampleMacroAnswer.tex}
  % %%%%%%%%%%%%%%%%%%%%%%%%%%%%%%%%%%%%%%%%%%%%%%%%%% 
  % Macro \solution
  % %%%%%%%%%%%%%%%%%%%%%%%%%%%%%%%%%%%%%%%%%%%%%%%%%% 
\item With the macro \verb|\solution|, we write the solution of an exercise.
  \examplerow{exampleMacroSolution.tex}
  % %%%%%%%%%%%%%%%%%%%%%%%%%%%%%%%%%%%%%%%%%%%%%%%%%% 
  % Macro \points
  % %%%%%%%%%%%%%%%%%%%%%%%%%%%%%%%%%%%%%%%%%%%%%%%%%% 
\item Set points to exercises with the macro \verb|\points|:
  \examplerow{exampleMacroPoints.tex}
  % %%%%%%%%%%%%%%%%%%%%%%%%%%%%%%%%%%%%%%%%%%%%%%%%%% 
  % Environment 'question'
  % %%%%%%%%%%%%%%%%%%%%%%%%%%%%%%%%%%%%%%%%%%%%%%%%%% 
\item Environment \verb|question|:
  \examplerow{exampleEnvirQuestion.tex}
  % %%%%%%%%%%%%%%%%%%%%%%%%%%%%%%%%%%%%%%%%%%%%%%%%%% 
  % Macro \hint
  % %%%%%%%%%%%%%%%%%%%%%%%%%%%%%%%%%%%%%%%%%%%%%%%%%% 
\item Hints with the macro \verb|\hint|:
  \examplerow{exampleMacroHint.tex}
\item Environment \verb|multichoice| is for multiple choice questions:
  % leave a blank line
  \examplerow{exampleEnvirMultichoice01.tex}

  Another example
  \examplerow{exampleEnvirMultichoice02.tex}
\item Environment \texttt{tickchoice}. Horizontal
  \examplerow{exampleEnvirTickchoice01.tex}

  and vertical

  \examplerow{exampleEnvirTickchoice02.tex}
\item A wish for good luck
  \examplerow{exampleMacroWish01.tex}

  Setting the text. Macro \verb/\letterspace/ sets the space between adjucent letters
    \examplerow{exampleMacroWish02.tex}

  \item Write the name and date:
    \examplerow{exampleMacroNameDate01.tex}

    Also, with dots or a line for blank space:
    \examplerow{exampleMacroNameDate02.tex}

    We could use
    \examplerow{exampleMacroNameDate03.tex}
  \item Exercise deadline:
        \examplerow{exampleMacroDeadline.tex}

      \item Set the duration of a test:
        \examplerow{exampleMacroDuration.tex}

      \item Add a remark in a document:
        \examplerow{exampleMacroRemark.tex}

      \item Add a reminder in a document:
        \examplerow{exampleMacroReminder.tex}

      \item Header for the theory part of a test:
        \examplerow{exampleMacroTheorypart.tex}

        Header for the exercise part of a test:
        \examplerow{exampleMacroExercisepart.tex}

      \item Set the title of a worksheet
        \examplerow{exampleMacroWorksheethd.tex}

      \item Teacher/headmaster signatures:
        \examplerow{exampleMacroSignatures.tex}

      \item Headers for tests:
        \examplerow{exampleMacroExamhd.tex}

      \item Header for end year summative tests:
        \examplerow{exampleMacroFinalExamhd.tex}

      \item School logo
        \examplerow{exampleMacroSchoollogo.tex}

      \item True-false type questions with the environment \verb|truefalse|
        \examplerow{exampleEnvirTruefalse01.tex}

      \item \verb|truefalse*| is a variant of \verb|truefalse|.
        \examplerow{exampleEnvirTruefalse02.tex}

      \item Matching questions:
        \examplerow{exampleMacroMatchingque.tex}
\end{enumerate}
\end{document}