\documentclass[10pt,a4page]{article}
\usepackage{xunicode}
\usepackage{xltxtra}
\usepackage{xgreek}

\setmainfont[Mapping=tex-text]{Linux Biolinum O}

\usepackage[width=0.9\paperwidth,height=0.9\paperheight,centering]{geometry}
\usepackage{xcolor}
\usepackage{amsfonts}
\usepackage{amsmath}
\usepackage{graphicx}
\usepackage{bm}
\usepackage{enumerate}
\usepackage{listings}

\lstset{
  basicstyle=\footnotesize\ttfamily,
  columns=flexible,
  breaklines=true,
  frame=single,
  framerule=0pt,
  % backgroundcolor=\color{blue!10},
  extendedchars=true,
  numbers=none,
  language=TeX,
  aboveskip=\medskipamount,
  belowskip=-1\medskipamount
}

\pagestyle{empty}

% Load the package 'schl'.
\usepackage[english]{schl}

% Change the space between adjacent rows in a tabular environment.
\renewcommand{\arraystretch}{2.5}

\author{Tassos Tsesmetzis}
\title{An overview of the \texttt{schl} package}
\date{}

\def\cellwidth{0.4\textwidth}
% Insert code and text examples in tables:
\newcommand{\miniexample}[3][t]{%
  \parbox[#1][#3][t]{\cellwidth}{#2}
}
\newcommand{\codeexample}[3][c]{%
  \colorbox[HTML]{b0c4be}{\miniexample[#1]{\lstinputlisting{#2}}{#3}}
}
\newcommand{\textexample}[3][c]{%
  {\footnotesize
    \colorbox[gray]{0.9}{\miniexample[#1]{\input{#2}}{#3}}%
  }
}

\newcommand{\examplerow}[3][10pt]{%
  \par\noindent\strut\hfill\codeexample{#2}{#3}\hspace{#1}\textexample{#2}{#3}\hfill\strut
}

% Height of examples. To be used as parameter to \examplerow
\newlength{\exheight}

\begin{document}
\maketitle

\verb|schl| is a \XeLaTeX\, package that provides commands and environments suitable for document types that appear in a classroom enviromnent. It's development is based on the Greek school educational practice, but it may be usefull in other contexts also. This document offers a quick view of working examples for \verb|schl|'s marcos. If we load the package passing the parameter \verb|greek|, several macros will be printed in Greek. These are
defined in \verb|languages/sch-greek.def|. If you want to set them in a different language modify the \verb|languages/sch-template.def| file.

\begin{enumerate}
\item Blank space is designated with the macros \verb|\lowerdots| and \verb|\blankspace|.
  % 
  \setlength{\exheight}{18pt}
  \lstset{
    aboveskip=1pt
  }
  \examplerow{exampleMacroBlankspace01.tex}{\exheight}
  % 
  \setlength{\exheight}{19pt}
  \examplerow{exampleMacroBlankspace02.tex}{\exheight}
  % 
  \setlength{\exheight}{19pt}
  \examplerow{exampleMacroBlankspace03.tex}{\exheight}
  % 
  \setlength{\exheight}{28pt}
  \examplerow{exampleMacroBlankspace04.tex}{\exheight}

  % %%%%%%%%%%%%%%%%%%%%%%%%%%%%%%%%%%%%%%%%%%%%%%%%%% 
  % Environment 'exercise'
  % %%%%%%%%%%%%%%%%%%%%%%%%%%%%%%%%%%%%%%%%%%%%%%%%%% 
\item With the environment \texttt{exercise} you can typeset exercises.
  % 
  \setlength{\exheight}{132pt}
  \examplerow{exampleEnvirExercise.tex}{\exheight}
  % %%%%%%%%%%%%%%%%%%%%%%%%%%%%%%%%%%%%%%%%%%%%%%%%%% 
  % Environment schltask'
  % %%%%%%%%%%%%%%%%%%%%%%%%%%%%%%%%%%%%%%%%%%%%%%%%%% 
\item The environment \texttt{schltask} can be used for summative tests.
  % 
  \setlength{\exheight}{57pt}
  \examplerow{exampleEnvirSchltask.tex}{\exheight}
  % %%%%%%%%%%%%%%%%%%%%%%%%%%%%%%%%%%%%%%%%%%%%%%%%%% 
  % Macro \asnswer
  % %%%%%%%%%%%%%%%%%%%%%%%%%%%%%%%%%%%%%%%%%%%%%%%%%% 
\item The macro \verb/\answer/ is used to typeset the answer of an exercise.
  % 
  \setlength{\exheight}{38pt}
  \examplerow{exampleMacroAnswer.tex}{\exheight}
  % %%%%%%%%%%%%%%%%%%%%%%%%%%%%%%%%%%%%%%%%%%%%%%%%%% 
  % Macro \solution
  % %%%%%%%%%%%%%%%%%%%%%%%%%%%%%%%%%%%%%%%%%%%%%%%%%% 
\item With the macro \verb|\solution|, we write the solution of an exercise.
  % 
  \setlength{\exheight}{56pt}
  \examplerow{exampleMacroSolution.tex}{\exheight}
  % %%%%%%%%%%%%%%%%%%%%%%%%%%%%%%%%%%%%%%%%%%%%%%%%%% 
  % Macro \points
  % %%%%%%%%%%%%%%%%%%%%%%%%%%%%%%%%%%%%%%%%%%%%%%%%%% 
\item Set points to exercises with the macro \verb|\points|:
  % 
  \setlength{\exheight}{113pt}
  \examplerow{exampleMacroPoints.tex}{\exheight}
  % %%%%%%%%%%%%%%%%%%%%%%%%%%%%%%%%%%%%%%%%%%%%%%%%%% 
  % Environment 'question'
  % %%%%%%%%%%%%%%%%%%%%%%%%%%%%%%%%%%%%%%%%%%%%%%%%%% 
\item Environment \verb|question|:
  % 
  \setlength{\exheight}{38pt}
  \examplerow{exampleEnvirQuestion.tex}{\exheight}
  % %%%%%%%%%%%%%%%%%%%%%%%%%%%%%%%%%%%%%%%%%%%%%%%%%% 
  % Macro \hint
  % %%%%%%%%%%%%%%%%%%%%%%%%%%%%%%%%%%%%%%%%%%%%%%%%%% 
\item Hints with the macro \verb|\hint|:
  % 
  \setlength{\exheight}{123pt}
  \examplerow{exampleMacroHint.tex}{\exheight}
\item Environment \verb|multichoice| is for multiple choice questions:
  % 
  \setlength{\exheight}{48pt}
  \examplerow{exampleEnvirMultichoice01.tex}{\exheight}

  Another example
  % 
  \setlength{\exheight}{57pt}
  \examplerow{exampleEnvirMultichoice02.tex}{\exheight}
\item Environment \texttt{tickchoice}. Horizontal
  % 
  \setlength{\exheight}{48pt}
  \examplerow{exampleEnvirTickchoice01.tex}{\exheight}

  and vertical

  % 
  \setlength{\exheight}{47pt}
  \examplerow{exampleEnvirTickchoice02.tex}{\exheight}
\item A wish for good luck
  % 
  \setlength{\exheight}{24pt}
  \examplerow{exampleMacroWish01.tex}{\exheight}

  Setting the text. Macro \verb/\letterspace/ sets the space between adjucent letters
  % 
  \setlength{\exheight}{48pt}
  \examplerow{exampleMacroWish02.tex}{\exheight}

\item Write the name and date:
  % 
  \setlength{\exheight}{20pt}
  \examplerow{exampleMacroNameDate01.tex}{\exheight}

  Also, with dots or a line for blank space:
  \examplerow{exampleMacroNameDate02.tex}{\exheight}

  We could use
  \examplerow{exampleMacroNameDate03.tex}{\exheight}
\item Exercise deadline:
  % 
  \setlength{\exheight}{10pt}
  \examplerow{exampleMacroDeadline.tex}{\exheight}

\item Set the duration of a test:
  % 
  \setlength{\exheight}{26pt}
  \examplerow{exampleMacroDuration.tex}{\exheight}

\item Add a remark in a document:
  % 
  \setlength{\exheight}{29pt}
  \examplerow{exampleMacroRemark.tex}{\exheight}

\item Add a reminder in a document:
  % 
  \setlength{\exheight}{19pt}
  \examplerow{exampleMacroReminder.tex}{\exheight}

\item Header for the theory part of a document:
  % 
  \setlength{\exheight}{27pt}
  \examplerow{exampleMacroTheorypart.tex}{\exheight}

  Header for the exercise part of a document:
  \examplerow{exampleMacroExercisepart.tex}{\exheight}

\item Set the title of a worksheet
  % 
  \setlength{\exheight}{47pt}
  \examplerow{exampleMacroWorksheethd.tex}{\exheight}

\item Teacher/headmaster signatures:
  % 
  \setlength{\exheight}{57pt}
  \setlength{\signatureslength}{80pt}
  \setlength{\signaturelineskip}{20pt}
  \examplerow{exampleMacroSignatures.tex}{\exheight}

\item Headers for tests:
  \examplerow{exampleMacroExamhd.tex}{\exheight}

\item Header for end year summative tests:
  % 
  \setlength{\exheight}{30pt}
  \examplerow{exampleMacroFinalExamhd.tex}{\exheight}

\item School logo
  % 
  \setlength{\exheight}{48pt}
  \examplerow{exampleMacroSchoollogo.tex}{\exheight}

\item True-false type questions with the environment \verb|truefalse|\\
  % 
  \setlength{\exheight}{68pt}
  \setlength{\truefalselength}{20pt}
  \examplerow{exampleEnvirTruefalse01.tex}{\exheight}

\item \verb|truefalse*| is a variant of \verb|truefalse|.
  % 
  \setlength{\exheight}{68pt}
  \examplerow{exampleEnvirTruefalse02.tex}{\exheight}

\item Matching questions:
  % 
  \setlength{\exheight}{98pt}
  \examplerow{exampleMacroMatchingque.tex}{\exheight}
\end{enumerate}
\end{document}