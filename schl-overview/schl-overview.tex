\documentclass[12pt,a4page]{article}
\usepackage{xunicode}
\usepackage{xltxtra}
\usepackage{xgreek}

\setmainfont[Mapping=tex-text]{Linux Biolinum O}

\usepackage[width=0.9\paperwidth,height=0.8\paperheight,centering]{geometry}
\usepackage{xcolor}
\usepackage{amsfonts}
\usepackage{amsmath}
\usepackage{graphicx}
\usepackage{bm}
\usepackage{enumerate}
\usepackage{listings}

\lstset{
  basicstyle=\footnotesize\ttfamily,
  columns=flexible,
  breaklines=true,
  frame=single,
  framerule=0pt,
  extendedchars=true,
  numbers=none,
  language=TeX,
  aboveskip=2pt,
  aboveskip=\medskipamount,
  belowskip=-1\medskipamount
}

% Load the package 'schl'.
\usepackage[english]{schl}

% Change the space between adjacent rows in a tabular environment.
\renewcommand{\arraystretch}{2.5}

\author{\small Tassos Tsesmetzis}
\title{%
  \vspace{-5ex}
  An overview of the  package \texttt{schl}%
  \vspace{-4ex}
}
\date{}

\def\cellwidth{0.4\textwidth}
% Insert code and text examples in tables:
\newcommand{\miniexample}[3][t]{%
  \parbox[#1][#3][t]{\cellwidth}{#2}
}
\newcommand{\codeexample}[3][c]{%
  \colorbox[HTML]{b0c4be}{\miniexample[#1]{\lstinputlisting{#2}}{#3}}
}
\newcommand{\textexample}[3][c]{%
  {\footnotesize
    \colorbox[gray]{0.9}{\miniexample[#1]{\input{#2}}{#3}}%
  }
}

\newcommand{\examplerow}[3][10pt]{%
  \par\noindent\strut\hfill\codeexample{#2}{#3}\hspace{#1}\textexample{#2}{#3}\hfill\strut
}

% Height of examples. To be used as parameter to \examplerow
\newlength{\exheight}

\begin{document}
\maketitle

\verb|schl| is a \XeLaTeX\, package that provides commands and environments suitable for document types that appear in a classroom enviromnent. It's development is based on the Greek school educational practice, but it may be usefull in other contexts also. This document offers a quick view of working examples for \verb|schl|'s marcos. 

We load the package with the option \verb|english|:

\begin{center}
\verb|\usepackage[english]{schl}|
\end{center}

so that several macros be printed in English. These are
defined in \verb|languages/schl-english.def|. Currently, the package
supports the options \verb|english| and \verb|greek|. Option \verb|greek| is more complete.

To set the macros of the package in a different language, start by copying the file \verb|languages/schl-template.def|. Rename it as \verb|languages/schl-<other-language>.def|. Then, set the macros and modify accordingly the \verb|schl.dtx| file.

The following list provides some basic example use cases.

\begin{enumerate}
  % %%%%%%%%%%%%%%%%%%%%%%%%%%%%%%%%%%%%%%%%%%%%%%%%%% 
  % Macros 'lowerdots' and '\blankspace'
  % %%%%%%%%%%%%%%%%%%%%%%%%%%%%%%%%%%%%%%%%%%%%%%%%%% 
\item Blank space is designated with the macros \verb|\lowerdots| and \verb|\blankspace|.
  % 
  \setlength{\exheight}{24pt}
  \examplerow{exampleMacroBlankspace01.tex}{\exheight}
  % 
  \setlength{\exheight}{24pt}
  \examplerow{exampleMacroBlankspace02.tex}{\exheight}
  % 
  \setlength{\exheight}{36pt}
  \examplerow{exampleMacroBlankspace03.tex}{\exheight}
  % 
  \setlength{\exheight}{38pt}
  \examplerow{exampleMacroBlankspace04.tex}{\exheight}
  % %%%%%%%%%%%%%%%%%%%%%%%%%%%%%%%%%%%%%%%%%%%%%%%%%% 
  % Macro '\wish'
  % %%%%%%%%%%%%%%%%%%%%%%%%%%%%%%%%%%%%%%%%%%%%%%%%%% 
\item Type a wish for good luck with the macro \verb|\wish|:
  % 
  \setlength{\exheight}{30pt}
  \examplerow{exampleMacroWish01.tex}{\exheight}

  We can change the text by redefining \verb|\wish|. The macro \verb/\letterspace/ sets the space between adjucent letters.
  % 
  \setlength{\exheight}{60pt}
  \examplerow{exampleMacroWish02.tex}{\exheight}
  % %%%%%%%%%%%%%%%%%%%%%%%%%%%%%%%%%%%%%%%%%%%%%%%%%% 
  % Macros '\fullname' and '\datefield'
  % %%%%%%%%%%%%%%%%%%%%%%%%%%%%%%%%%%%%%%%%%%%%%%%%%% 
\item With the macros \verb|\fullname| and \verb|\datefield| we write the name and date:
  % 
  \setlength{\exheight}{24pt}
  \examplerow{exampleMacroNameDate01.tex}{\exheight}

  Also, with dots or a line for blank space:
  \examplerow{exampleMacroNameDate02.tex}{\exheight}

  We can pass a date using the macros \verb|\setdate| and \verb|\getdate|.
  \examplerow{exampleMacroNameDate03.tex}{\exheight}
  % %%%%%%%%%%%%%%%%%%%%%%%%%%%%%%%%%%%%%%%%%%%%%%%%%% 
  % Macro '\deadline'
  % %%%%%%%%%%%%%%%%%%%%%%%%%%%%%%%%%%%%%%%%%%%%%%%%%% 
\item Write a deadline with the macro \verb|\deadline|
  % 
  \setlength{\exheight}{12pt}
  \examplerow{exampleMacroDeadline.tex}{\exheight}
  % %%%%%%%%%%%%%%%%%%%%%%%%%%%%%%%%%%%%%%%%%%%%%%%%%% 
  % Macro '\duration'
  % %%%%%%%%%%%%%%%%%%%%%%%%%%%%%%%%%%%%%%%%%%%%%%%%%%
\item Set the duration of a test with \verb|\duration|
  % 
  \setlength{\exheight}{36pt}
  \examplerow{exampleMacroDuration.tex}{\exheight}
  % %%%%%%%%%%%%%%%%%%%%%%%%%%%%%%%%%%%%%%%%%%%%%%%%%% 
  % Macro '\remark'
  % %%%%%%%%%%%%%%%%%%%%%%%%%%%%%%%%%%%%%%%%%%%%%%%%%%
\item Add a remark in a document with \verb|\remark|
  % 
  \setlength{\exheight}{36pt}
  \examplerow{exampleMacroRemark.tex}{\exheight}
  % %%%%%%%%%%%%%%%%%%%%%%%%%%%%%%%%%%%%%%%%%%%%%%%%%% 
  % Macro '\remark'
  % %%%%%%%%%%%%%%%%%%%%%%%%%%%%%%%%%%%%%%%%%%%%%%%%%%
\item Add a reminder with \verb|\reminder|:
  % 
  \setlength{\exheight}{24pt}
  \examplerow{exampleMacroReminder.tex}{\exheight}
  % %%%%%%%%%%%%%%%%%%%%%%%%%%%%%%%%%%%%%%%%%%%%%%%%%% 
  % Macros '\theorypart' and '\exercise' part
  % %%%%%%%%%%%%%%%%%%%%%%%%%%%%%%%%%%%%%%%%%%%%%%%%%%
\item Add a header for the theory part of a document with \verb|\theorypart|
  % 
  \setlength{\exheight}{32pt}
  \examplerow{exampleMacroTheorypart.tex}{\exheight}

  Add a header for the exercise part of a document with \verb|\exercisepart|
  \examplerow{exampleMacroExercisepart.tex}{\exheight}

  \newpage

  % %%%%%%%%%%%%%%%%%%%%%%%%%%%%%%%%%%%%%%%%%%%%%%%%%% 
  % Macro '\matchingque'
  % %%%%%%%%%%%%%%%%%%%%%%%%%%%%%%%%%%%%%%%%%%%%%%%%%% 
\item The macro \verb|\matchingque| can be used for matching questions
  % 
  \setlength{\exheight}{117pt}
  \setlength{\leftmatchwidth}{10em}
  \setlength{\rightmatchwidth}{10em}
  \examplerow{exampleMacroMatchingque.tex}{\exheight}
  % %%%%%%%%%%%%%%%%%%%%%%%%%%%%%%%%%%%%%%%%%%%%%%%%%% 
  % Macro \asnswer
  % %%%%%%%%%%%%%%%%%%%%%%%%%%%%%%%%%%%%%%%%%%%%%%%%%% 
\item The macro \verb/\answer/ is used to typeset the answer of an exercise.
  % 
  \setlength{\exheight}{49pt}
  \examplerow{exampleMacroAnswer.tex}{\exheight}
  % %%%%%%%%%%%%%%%%%%%%%%%%%%%%%%%%%%%%%%%%%%%%%%%%%% 
  % Macro \solution
  % %%%%%%%%%%%%%%%%%%%%%%%%%%%%%%%%%%%%%%%%%%%%%%%%%% 
\item With the macro \verb|\solution|, we write the solution of an exercise.
  % 
  \setlength{\exheight}{98pt}
  \examplerow{exampleMacroSolution.tex}{\exheight}
  % %%%%%%%%%%%%%%%%%%%%%%%%%%%%%%%%%%%%%%%%%%%%%%%%%% 
  % Macro \points
  % %%%%%%%%%%%%%%%%%%%%%%%%%%%%%%%%%%%%%%%%%%%%%%%%%% 
\item Set points to exercises with the macro \verb|\points|:
  % 
  \setlength{\exheight}{169pt}
  \examplerow{exampleMacroPoints.tex}{\exheight}

  \newpage

  % %%%%%%%%%%%%%%%%%%%%%%%%%%%%%%%%%%%%%%%%%%%%%%%%%% 
  % Macro \hint
  % %%%%%%%%%%%%%%%%%%%%%%%%%%%%%%%%%%%%%%%%%%%%%%%%%% 
\item Write hints with the macro \verb|\hint|:
  % 
  \setlength{\exheight}{180pt}
  \examplerow{exampleMacroHint.tex}{\exheight}
  % %%%%%%%%%%%%%%%%%%%%%%%%%%%%%%%%%%%%%%%%%%%%%%%%%% 
  % Macros '\worksheethd'
  % %%%%%%%%%%%%%%%%%%%%%%%%%%%%%%%%%%%%%%%%%%%%%%%%%% 
\item Set the title of a worksheet with \verb|\worksheethd|
  % 
  \setlength{\exheight}{70pt}
  \examplerow{exampleMacroWorksheethd.tex}{\exheight}
  % %%%%%%%%%%%%%%%%%%%%%%%%%%%%%%%%%%%%%%%%%%%%%%%%%% 
  % Macro '\signatures'
  % %%%%%%%%%%%%%%%%%%%%%%%%%%%%%%%%%%%%%%%%%%%%%%%%%% 
\item Designate space for teacher(s)/headmaster signatures with the macro
  \verb|\signatures|
  % 
  \setlength{\exheight}{72pt}
  \setlength{\signatureslength}{80pt}
  \setlength{\signaturelineskip}{25pt}
  \examplerow{exampleMacroSignatures.tex}{\exheight}
  % %%%%%%%%%%%%%%%%%%%%%%%%%%%%%%%%%%%%%%%%%%%%%%%%%% 
  % Macro '\examhd'
  % %%%%%%%%%%%%%%%%%%%%%%%%%%%%%%%%%%%%%%%%%%%%%%%%%% 
\item Headers for tests can be set with the macro \verb|\examhd|
  \setlength{\exheight}{67pt}
  \examplerow{exampleMacroExamhd.tex}{\exheight}
  % %%%%%%%%%%%%%%%%%%%%%%%%%%%%%%%%%%%%%%%%%%%%%%%%%% 
  % Macro '\finalexamhd'
  % %%%%%%%%%%%%%%%%%%%%%%%%%%%%%%%%%%%%%%%%%%%%%%%%%% 
\item With \verb|\finalexamhd| we can set a header for end year summative tests.
  % 
  \setlength{\exheight}{39pt}
  \examplerow{exampleMacroFinalExamhd.tex}{\exheight}

  \newpage

  % %%%%%%%%%%%%%%%%%%%%%%%%%%%%%%%%%%%%%%%%%%%%%%%%%% 
  % Macro '\schoollogo'
  % %%%%%%%%%%%%%%%%%%%%%%%%%%%%%%%%%%%%%%%%%%%%%%%%%% 
\item A school logo can be set with \verb|\schoollogo|.
  % 
  \setlength{\exheight}{62pt}
  \examplerow{exampleMacroSchoollogo.tex}{\exheight}
  % %%%%%%%%%%%%%%%%%%%%%%%%%%%%%%%%%%%%%%%%%%%%%%%%%% 
  % Environment 'truefalse'
  % %%%%%%%%%%%%%%%%%%%%%%%%%%%%%%%%%%%%%%%%%%%%%%%%%% 
\item True-false type questions can be set with the environment \verb|truefalse|\\
  % 
  \setlength{\exheight}{85pt}
  \setlength{\truefalselength}{20pt}
  \examplerow{exampleEnvirTruefalse01.tex}{\exheight}
  % %%%%%%%%%%%%%%%%%%%%%%%%%%%%%%%%%%%%%%%%%%%%%%%%%% 
  % Environment 'truefalse*'
  % %%%%%%%%%%%%%%%%%%%%%%%%%%%%%%%%%%%%%%%%%%%%%%%%%% 
\noindent \verb|truefalse*| is a variant of \verb|truefalse|:
  % 
\setlength{\exheight}{95pt}
\setlength{\truefalselength}{30pt} 
\examplerow{exampleEnvirTruefalse02.tex}{\exheight}
  % %%%%%%%%%%%%%%%%%%%%%%%%%%%%%%%%%%%%%%%%%%%%%%%%%% 
  % Environment 'exercise'
  % %%%%%%%%%%%%%%%%%%%%%%%%%%%%%%%%%%%%%%%%%%%%%%%%%% 
\item With the environment \texttt{exercise} you can typeset exercises.
  % 
  \setlength{\exheight}{193pt}
  \examplerow{exampleEnvirExercise.tex}{\exheight}

  \newpage

  % %%%%%%%%%%%%%%%%%%%%%%%%%%%%%%%%%%%%%%%%%%%%%%%%%% 
  % Environment schltask'
  % %%%%%%%%%%%%%%%%%%%%%%%%%%%%%%%%%%%%%%%%%%%%%%%%%% 
\item The environment \texttt{schltask} can be used for summative tests.
  % 
  \setlength{\exheight}{86pt}
  \examplerow{exampleEnvirSchltask.tex}{\exheight}
  % %%%%%%%%%%%%%%%%%%%%%%%%%%%%%%%%%%%%%%%%%%%%%%%%%% 
  % Environment 'question'
  % %%%%%%%%%%%%%%%%%%%%%%%%%%%%%%%%%%%%%%%%%%%%%%%%%% 
\item The environment \verb|question| can be used to typeset a list of questions.
  % 
  \setlength{\exheight}{61pt}
  \examplerow{exampleEnvirQuestion.tex}{\exheight}
  % %%%%%%%%%%%%%%%%%%%%%%%%%%%%%%%%%%%%%%%%%%%%%%%%%% 
  % Environment 'multichoice'
  % %%%%%%%%%%%%%%%%%%%%%%%%%%%%%%%%%%%%%%%%%%%%%%%%%% 
\item The environment \verb|multichoice| is for multiple choice questions:
  % 
  \setlength{\exheight}{72pt}
  \examplerow{exampleEnvirMultichoice01.tex}{\exheight}

  Another example
  % 
  \setlength{\exheight}{85pt}
  \examplerow{exampleEnvirMultichoice02.tex}{\exheight}
  % %%%%%%%%%%%%%%%%%%%%%%%%%%%%%%%%%%%%%%%%%%%%%%%%%% 
  % Environment 'tickchoice'
  % %%%%%%%%%%%%%%%%%%%%%%%%%%%%%%%%%%%%%%%%%%%%%%%%%% 
\item Environment \texttt{tickchoice}. Horizontal alignment
  % 
  \setlength{\exheight}{60pt}
  \examplerow{exampleEnvirTickchoice01.tex}{\exheight}

  and vertical

  % 
  \setlength{\exheight}{60pt}
  \examplerow{exampleEnvirTickchoice02.tex}{\exheight}
\end{enumerate}
\end{document}
