% %%%%%%%%%%%%%%%%%%%%%%%%%%%%%%%%%%%%%%%%%%%%%%%%%%%%%%%%%%%%%%%%%%%%%% 
% Examples for the xelatex package schl
% Tassos Tsesmetzis
% 
% %%%%%%%%%%%%%%%%%%%%%%%%%%%%%%%%%%%%%%%%%%%%%%%%%%%%%%%%%%%%%%%%%%%%%% 
\documentclass[12pt,a4page]{article}
\usepackage{fontspec}
\usepackage{xunicode}
\usepackage{xltxtra}
\usepackage{xgreek}

\setmainfont[Mapping=tex-text]{Linux Biolinum O}

% Package geometry. Set text area dimensions.
\usepackage[width=0.9\paperwidth,height=0.95\paperheight,centering]{geometry}
% Package amsfonts. We need it for the macro \mathbb.
\usepackage{amsfonts}
% Package enumerate. 
\usepackage{enumerate}
% Package tikz for graphics.
\usepackage{tikz}
\usetikzlibrary{calc,math}
% Load package schl.
\usepackage[greek]{schl}
% Load the gr-math package
\usepackage{gr-math}

% Don't print page number
\pagestyle{empty}

% Macros for this document 
% \lowerdots deviation from the base line.
\def\dotDeviation{-0.5ex}
% \lowerdots: Number of dots for filling in a word
\def\dotNoForWord{35}

% Set \school.
\school{Γυμνάσιο Ξάνθης}
% Set name of the teacher.
\teacher{Όνομα εκπαιδευτικού}
% Set suject.
\subject{Μαθηματικά}
% Set grade.
\grade{Α' Γυμνασίου}

% A new macro for blank space to avoid some typing.
\def\blank{\lowerdots[0ex]{10}}

% Print text in multichoice environment in \small size.
\setlist*[multichoice]{before=\small}
% Set vertical distance between adjucent exercises to 7ex.
\setlist*[exercise]{itemsep=7ex}
% Print the question number in the true-false type questions in arabic
% and set question to question vertical distance to 5ex.
\setlist*[truefalse]{label=\arabic*.,itemsep=5ex,leftmargin=14pt}
% The next four lines are for the macro \matchingque.
\setlist*[leftmatching]{label=,itemsep=5ex}
\setlist*[rightmatching]{label=,itemsep=2.5ex}
\setlength{\rightmatchwidth}{80pt}
\setlength{\rightmatchwidth}{180pt}

\begin{document}
% Print school logo.
\noindent\schoollogo{200pt}
\hfill
% Print date at the end of line.
\datefield{16/01/2019}

\vspace{1ex}

% Set the header.
\examhd[Επαναληπτικό τεστ στη Γεωμετρία]{}

\vspace{4ex}

\fullname{\lowerdots{60}}

\vspace{1ex}

% Enlarge the space between adjucent lines of text.
% We want our student's handwritten letters to fit in the blank spaces.
\setlength\baselineskip{3.55ex}

\begin{exercise}
  % %%%%%%%%%%%%%%%%%%%%%%%%%%%%%%%%%%%%%%%%%%%%%%%%%% 
  % First exercise
  % %%%%%%%%%%%%%%%%%%%%%%%%%%%%%%%%%%%%%%%%%%%%%%%%%% 
\item Να συμπληρώσετε τα κενά επιλέγοντας μία από τις απαντήσεις που δίνονται:\points{6}

  \vspace{-1.5ex}

  \begin{enumerate}[label=\arabic*.,leftmargin=14pt,itemsep=2ex]
  \item Από δύο σημεία \blank

    % One could also use the more powerfull package tasks for
    % typing multiple choice questions.
    \begin{multichoice}[itemjoin=\hspace{4em}]
    \item μπορούν να περάσουν άπειρες ευθείες
    \item μπορεί να περάσει μόνο μία ευθεία\\\hspace*{9em}
      % The three questions don't fit in a line.
      % Print the third question in the next line.
      % Also, add some horizontal space before third item.
      % This space is determined with trial and error. Any other idea?
    \item μπορούν να περάσουν μόνο δύο ευθείες
    \end{multichoice}
  \item Αν οι πλευρές μίας γωνίας είναι ημιευθείες κάθετες μεταξύ τους, τότε η γωνία λέγεται \blank
    
    \begin{multichoice}
    \item οξεία
    \item ορθή
    \item αμβλεία
    \end{multichoice}
  \item Δύο ευθείες που βρίσκονται στο ίδιο επίπεδο και δεν έχουν κανένα κοινό σημείο, λέγονται \blank
    
    \begin{multichoice}
    \item παράλληλες
    \item τεμνόμενες
    \item κάθετες
    \end{multichoice}
  \item Δύο γωνίες που έχουν άθροισμα $180^\circ$ ονομάζονται \blank\ γωνίες.
    
    \begin{multichoice}
    \item κατακορυφήν
    \item συμπληρωματικές
    \item παραπληρωματικές
    \end{multichoice}
  \item Δύο γωνίες που έχουν άθροισμα $90^\circ$ ονομάζονται \blank\ γωνίες.
    
    \begin{multichoice}
    \item κατακορυφήν
    \item συμπληρωματικές
    \item παραπληρωματικές
    \end{multichoice}
  \item Δύο γωνίες που έχουν την κορυφή τους κοινή και τις πλευρές τους αντικείμενες ημιευθείες ονομάζονται \blank\ γωνίες.
    
    \begin{multichoice}
    \item κατακορυφήν
    \item συμπληρωματικές
    \item παραπληρωματικές
    \end{multichoice}
  \end{enumerate}
  % %%%%%%%%%%%%%%%%%%%%%%%%%%%%%%%%%%%%%%%%%%%%%%%%%% 
  % Second exercise
  % %%%%%%%%%%%%%%%%%%%%%%%%%%%%%%%%%%%%%%%%%%%%%%%%%% 
\item Να χαρακτηρίσετε κάθε μία από τις πιο κάτω προτάσεις ως σωστή ή λάθος:\points{3}

  \vspace{-1.5ex}

  \begin{truefalse}
  \item Το μήκος της τεθλασμένης γραμμής $AB\Gamma\Delta$ είναι $5\,cm$.\\
    
    \hspace{12em}
    % One would prefer to enclose the next tikzpicture inside a
    % \begin{center} ... \end{center}
    % instead of prepending a 12em space.
    % Unfortunately, this doesn't work due to the way that
    % the \item is redefined inside truefalse.
    \begin{tikzpicture}
      \coordinate (A) at (-120:2);
      \coordinate (B) at (0,0);
      \coordinate (C) at (-30:2);
      \coordinate (D) at ($(C) + (10:3)$);
      \draw (A) node [below] {\small$A$} -- (B) node [above] {\small$B$}%
      -- (C) node [below] {\small$\Gamma$} -- (D) node [above] {\small$\Delta$};
      \node [above left] at ($0.5*(A) + 0.5*(B)$) {\small$2\ cm$};
      \node [above right] at ($0.5*(B) + 0.5*(C)$) {\small$2\ cm$};
      \node [below right] at ($0.5*(C) + 0.5*(D)$) {\small$3\ cm$};
    \end{tikzpicture}
  \item Το σημείο $M$ είναι το μέσο του τμήματος $AB$.\\
    
    \hspace{15em}
    \begin{tikzpicture}
      \coordinate (A) at (0,0);
      \coordinate (M) at (1.5,0);
      \coordinate (B) at (4,0);
      \draw [|-|]  (A) node [below] {\small$A$} -- (M) node [below] {\small$M$};
      \draw [-|] (M)-- (B) node [below] {\small$B$};
      \node [above] at ($0.5*(A) + 0.5*(M)$) {\small$1,\!5\ cm$};
      \node [above] at ($0.5*(B) + 0.5*(M)$) {\small$2,\!5\ cm$};
    \end{tikzpicture}
  \item  Οι ευθείες $\varepsilon_1$ και $\varepsilon_2$ είναι κάθετες.
    
    \hspace{16.3em}
    \begin{tikzpicture}
      \coordinate (A) at (1,1);
      \coordinate (B) at (1,-1);
      \draw (A) node [right] {\small$\varepsilon_1$} -- ($-1*(A)$);
      \draw (B) -- ($-1*(B)$) node [left] {\small$\varepsilon_2$};
      \fill [black] (45:0.2) -- (0,0) --(-45:0.2) -- (0.2 * sqrt 2,0) -- cycle;
    \end{tikzpicture}
  \end{truefalse}
  % %%%%%%%%%%%%%%%%%%%%%%%%%%%%%%%%%%%%%%%%%%%%%%%%%% 
  % Third exercise
  % %%%%%%%%%%%%%%%%%%%%%%%%%%%%%%%%%%%%%%%%%%%%%%%%%% 
\item Ενώστε κάθε γωνία με το είδος της.\points{6}

  \matchingque[350pt]{%
    % First angle
    \hspace{1.5em}
    \begin{tikzpicture}
      \draw (15:1) -- (0:0) node [above right] {\small$\varphi$} -- (80:1);
    \end{tikzpicture},%
    % Second angle
    \begin{tikzpicture}
      \draw (30:1) -- (0:0) node [above] {\small$\varphi$} -- (160:1);
    \end{tikzpicture},
    % Third angle
    \hspace{1.3em}
    \begin{tikzpicture}
      \draw (0:1) -- (0:0) node [below right] {\small$\varphi$} -- (-90:1);
      \fill (0:0) rectangle (-45:0.2);
    \end{tikzpicture}%
  }{%
    Η γωνία $\varphi$ είναι ευθεία γωνία.,%
    Η γωνία $\varphi$ είναι αμβλεία γωνία.,%
    Η γωνία $\varphi$ είναι ορθή γωνία.,%
    Η γωνία $\varphi$ είναι οξεία γωνία.%
  }
  % %%%%%%%%%%%%%%%%%%%%%%%%%%%%%%%%%%%%%%%%%%%%%%%%%% 
  % Forth exercise
  % %%%%%%%%%%%%%%%%%%%%%%%%%%%%%%%%%%%%%%%%%%%%%%%%%% 
\item Δίνεται ο κύκλος $(K\,,\, 2 cm)$.
  \begin{enumerate}[leftmargin=14pt]
  \item Σχεδιάστε μια διάμετρο $AB$.\points{1}
  \item Από τα σημεία $A,\ B$ σχεδιάστε τις εφαπτόμενες του κύκλου.\points{2}
  \item Ποια είναι η σχετική θέση των εφαπτομένων που σχεδιάσατε; Δικαιολογήστε την απάντησή σας.\\\phantom{.}\points{2}
  \end{enumerate}

  \vspace{4ex}
  
  \begin{center}
    \begin{tikzpicture}
      \coordinate (K) at (0,);
      \draw (K) circle [radius=2cm];
      \fill (K) node [above] {\small$K$} circle [radius=1pt];
    \end{tikzpicture}
  \end{center}
\end{exercise}
\end{document}
