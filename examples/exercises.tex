% %%%%%%%%%%%%%%%%%%%%%%%%%%%%%%%%%%%%%%%%%%%%%%%%%%%%%%%%%%%%%%%%%%%%%%
%                       Examples for the xelatex package schl
%                               Tassos Tsesmetzis
%
% %%%%%%%%%%%%%%%%%%%%%%%%%%%%%%%%%%%%%%%%%%%%%%%%%%%%%%%%%%%%%%%%%%%%%%
\documentclass[12pt,a4page]{article}
\usepackage{xunicode}
\usepackage{xltxtra}
\usepackage{xgreek}

\setmainfont[Mapping=tex-text]{Linux Biolinum O}

% Package geometry. Set text area dimensions.
\usepackage[width=0.9\paperwidth,height=0.9\paperheight,centering]{geometry}
% Package amsfonts. We need it for the macro \mathbb.
\usepackage{amsfonts}

% Load the package schl
\usepackage[greek]{schl}
% Load the gr-math package
\usepackage{gr-math}

% Don't print page number
\pagestyle{empty}

% Set \school.
\school{{\letterspace{\defaultletterspace} ΓΕΛ} Αθηνών}
% Set name of the teacher.
\teacher{Όνομα εκπαιδευτικού}
% Set suject.
\subject{Άλγεβρα}
% Set grade.
\grade{Β' Λυκείου}

\begin{document}
% Print school logo with size 40ex and no indentation.
\noindent\schoollogo{40ex}%
% Print deadline at the end of the current line.
\hfill\deadline{Δευτέρα, 4 Μαίου}

% Leave some space between header and logo.
\vspace{4ex}

\heading{Επαναληπτικές ασκήσεις}

% Leave some space between header and the exercises.
\vspace{1ex}

\begin{exercise}
\item Υπολογίστε την τιμή του πολυωνύμου $P(x) = x^2 - x + 1$, όταν $x = 1$.\\
  % Print answer at the end of line. We need \phantom{.} because there is no
  % letter in front of \answer.
  \phantom{.}\answer{$1$}
\item Να λύσετε την εξίσωση $\sin x = \cos x$.\\
  \phantom{.}\answer{$x = \frac\pi{4} + \kappa \pi,\, \kappa\in\mathbb{Z}$}
  % Add a hint to the exercise.
  \hint{Παρατηρήστε ότι η εξίσωση δεν έχει ρίζες τους αριθμούς
    $x = \frac{\pi}2 + \kappa \pi,\, \kappa\in\mathbb{Z}$. \\
    Έτσι, για $x \neq \frac{\pi}2 + \kappa \pi,\, \kappa\in\mathbb{Z}$, μπορούμε
    να γράψουμε $$\frac{\sin x}{\cos x} = 1 \Longleftrightarrow \tan x = 1
    \Longleftrightarrow \ldots$$
    }
  \item Να λύσετε την εξίσωση $$x^3 - 4x^2 + x + 6 =0$$
    \answer{$x=-1$ ή $x=2$ ή $x=3$}
    \hint{Αναζητήστε μια ρίζα $\rho$ στους ακέραιους διαιρέτες του σταθερού όρου $6$. Στη συνέχεια, υπολογίστε: $(x^3 - 4x^2 + x + 6):(x - \rho)$ και παραγοντοποιήστε το αριστερό μέλος της εξίσωσης.%
      }
    \item Βρείτε τις λύσεις της ανίσωσης
      $$e^{3x} \leq e^{2x + \ln 2} + e^x - 2$$
      \answer{$0\leq x\leq \ln 2$}
      \hint{Η ανίσωση γράφεται ισοδύναμα:
        $$e^{3x} \leq 2 e^{2x} + e^x - 2 \Longleftrightarrow e^{3x} - 2 e^{2x} - e^x + 2 \leq 0$$
        Θέστε $w = e^x$. Τότε η ανίσωση μετασχηματίζεται σε μια πολυωνυμική ανίσωση με μεταβλητή το $w$.
        }
      \item Για τις διάφορες τιμές του $\lambda\in\mathbb{R}$, να λύσετε το σύστημα
        $$\left\{
          \begin{array}{cl}
            \lambda  x  - 2 y & = 1 \\
            4x  + y & = -2
          \end{array}
        \right.
        $$
        \answer{%
          % Our answers splits in several lines. We use a \parbox to put all text in
          % a paragraph. First argument [c], selects as baseline of the text the horizontal
          % line through the midpoint of it's height. Second argument {150pt} is the
          % length of the box.
          \parbox[c]{150pt}{%
            % Choose the smallest size for our font.
            {\tiny
            \begin{itemize}
            \item Αν $\lambda \neq -8$ το σύστημα έχει μοναδική λύση την
              $(x\,,\, y) = (-\frac1{\lambda + 8}\,,\, -\frac{\lambda + 4}{\lambda + 8})$.
            \item Αν $\lambda = -8$ το σύστημα είναι αδύνατο.
            \end{itemize}
            }
          }
        }
        \hint{Υπολογίστε τις ορίζουσες του συστήματος $D,\ D_x,\ D_y$. Στη συνέχεια, βρείτε τις τιμές του
          $\lambda$ για τις οποίες $D = 0$. Διακρίνουμε τις περιπτώσεις
          \begin{itemize}
          \item Για εκείνα τα $\lambda\in\mathbb{R}$ για τα οποία $D \neq 0$, το σύστημα έχει μοναδική
            λύση την \ldots
          \item Για κάθε μία από τις ρίζες της εξίσωσης $D = 0$, το σύστημα είναι είτε αόριστο, είτε αδύνατο\ldots          \end{itemize}
        }
\end{exercise}
\end{document}