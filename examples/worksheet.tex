% %%%%%%%%%%%%%%%%%%%%%%%%%%%%%%%%%%%%%%%%%%%%%%%%%%%%%%%%%%%%%%%%%%%%%%
%                       Examples for the xelatex package schl
%                               Tassos Tsesmetzis
%
% %%%%%%%%%%%%%%%%%%%%%%%%%%%%%%%%%%%%%%%%%%%%%%%%%%%%%%%%%%%%%%%%%%%%%%
\documentclass[12pt,a4page]{article}
\usepackage{fontspec}
\usepackage{xunicode}
\usepackage{xltxtra}
\usepackage{xgreek}

\setmainfont[Mapping=tex-text]{Linux Biolinum O}

% Package geometry. Set text area dimensions.
\usepackage[width=0.9\paperwidth,height=0.95\paperheight,centering]{geometry}
% Package amsfonts. We need it for the macro \mathbb.
\usepackage{amsfonts}
% Package enumerate. 
\usepackage{enumerate}
% Package tikz for graphics.
\usepackage{tikz}
% Load package schl.
\usepackage[greek]{schl}
% Load the gr-math package
\usepackage{gr-math}

% Don't print page number
\pagestyle{empty}

% Macros for this document 
% \lowerdots deviation from the base line.
\def\dotDeviation{-0.5ex}
% \lowerdots: Number of dots for filling in a word
\def\dotNoForWord{35}

% Set \school.
\school{Γυμνάσιο Θεσσαλονίκης}
% Set name of the teacher.
\teacher{Όνομα εκπαιδευτικού}
% Set suject.
\subject{Μαθηματικά}
% Set grade.
\grade{Α' Γυμνασίου}


\begin{document}
% Print school logo.
\noindent\schoollogo{200pt}
\hfill
% Print date at the end of line.
% We use \makebox to leave a blank space after the Greek \date@term.
% Another idea might be to replace the \makebox macro with something like
% \phantom{<A test date>}.
\datefield{\makebox[6em]{}}

% Print a header for the worksheet appending the given argument to
% the Greek \worksheet@term.
\worksheethd{στην παράγραφο Α.7.1}

% Enlarge the space between adjucent lines of text.
% We want out student's handwritten letters to fit in the blank spaces.
\setlength\baselineskip{3.55ex}

% Exercises are adapted from unit A.7.1 of the Greek textbook:
% ΜΑΘΗΜΑΤΙΚΑ Α' ΓΥΜΝΑΣΙΟΥ
% Βανδουλάκης Ι, Καλλιγάς Χ, Μαρκάκης Ν, Φερεντίνος Σ
% ΟΕΔΒ
\begin{enumerate}[\bf\large 1.]
\item  Να συμπληρώσετε τα κενά:\\[-5ex]
  \begin{enumerate}[\bf (a)]
  \item Οι αριθμοί που έχουν πρόσημο $+$ λέγονται \lowerdots[\dotDeviation]{\dotNoForWord}, ενώ αυτοί που έχουν πρόσημο $-$ λέγονται \lowerdots[\dotDeviation]{\dotNoForWord}
  \item Οι αριθμοί με το ίδιο πρόσημο λέγονται \lowerdots[\dotDeviation]{\dotNoForWord}, ενώ αυτοί με διαφορετικά πρόσημα λέγονται \lowerdots[\dotDeviation]{\dotNoForWord}
  \item Στην ευθεία των αριθμών, δεξιά του μηδέν βρίσκονται οι \lowerdots[\dotDeviation]{\dotNoForWord} ρητοί και αριστερά του μηδέν οι \lowerdots[\dotDeviation]{\dotNoForWord} ρητοί.\\[-1ex]
    \begin{center}
      \begin{tikzpicture}
        \draw [->, >=stealth] (-5,0) -- (5,0);
        \foreach \i in {-4,-3,-2,-1,0,1,2,3,4}
                \draw [very thin] (\i,0.1) -- (\i,-0.1);
        \node at (0,-0.5) {\large $0$};
        \node at (5,-0.3) {\large $x$};
        \node at (-5,-0.3) {\large $x'$};
      \end{tikzpicture}
    \end{center}

    \vspace{-2.5ex}

  \item Φυσικοί αριθμοί: \lowerdots[\dotDeviation]{\dotNoForWord} \\
    \noindent Ακέραιοι αριθμοί: \lowerdots[\dotDeviation]{\dotNoForWord} \\
    \noindent Ρητοί αριθμοί: \lowerdots[\dotDeviation]{\dotNoForWord}
  \end{enumerate}
\item Να κατατάξετε τους παρακάτω αριθμούς σε δύο ομάδες, τους θετικούς και τους αρνητικούς:
  $$-3,\!1,\quad +5,\quad +\frac83,\quad -20,\!7,\quad 11,\quad -2, \quad 18,\!5$$
  % Print \solution@term. This area will be filled by students.
  \solution{%
    Θετικοί αριθμοί:\\
    Αρνητικοί  αριθμοί:
  }
\item Στα ζεύγη αριθμών που ακολουθούν, να βρείτε ποιοι αριθμοί είναι ομόσημοι και ποιοι ετερόσημοι:
  \begin{center}
    % Use the multichoice environment. With the argument label=\bf(\alph*) we
    % format the label of the environment.
    % See the enumitem package documentation for more details.
    \begin{multichoice}[label=\bf(\alph*)]
    \item $3$ και $+3$
    \item $-2$ και $-4$
    \item $3$ και $-1,\!2$
    \item $-\frac53$ και $+1,\!9$
    \item $0$ και $-5$
    \item $\frac12$ και $0$
    \end{multichoice}
  \end{center}
  % Another idea is to use a tabular environment and define
  % a counter, questionCounter, for each question. The macro
  % \questionNo advances the counter and formats it's value. 
  % \newcounter{questionCounter}
  % \setcounter{questionCounter}{0}
  % \def\questionNo{\stepcounter{questionCounter}(\alph{questionCounter})\quad}
  % \begin{center}
  %   \begin{tabular}{p{150pt}p{150pt}l}
  %     \questionNo $3$ και $+3$ & \questionNo $-2$ και $-4$ & \questionNo $3$ και $-1,\!2$ \\[1ex]
  %     \questionNo $-\frac53$ και $+1,\!9$ & \questionNo $0$ και $-5$ & \questionNo $\frac12$ και $0$
  %   \end{tabular}
  % \end{center}
\solution{%
  Ομόσημοι:\\
  Ετερόσημοι:\\
  Άλλο:
}
\item Τοποθετήστε στην ευθεία των αριθμών, τα σημεία με τετμημένη:
  $$0,\quad -1,\!5,\quad 4,\quad +\frac12,\quad -2,\quad -\frac72,\quad +3,\!8$$
  \solution{%

    \vspace{3ex}

    \begin{center}
      \begin{tikzpicture}
        \draw [->, >=stealth, line width=1pt] (-5,0) -- (5,0);
        \foreach \i in {-4.8,-4.7,...,4.8}
        \draw [line width=0.1pt] (\i,0.07) -- (\i,-0.07);
        \foreach \i in {-4.5,-4.0,...,4.5}
        \draw [line width=0.3pt] (\i,0.15) -- (\i,-0.15);
        \node at (5,-0.25) {\large $x$};
        \node at (-5,-0.25) {\large $x'$};
      \end{tikzpicture}
    \end{center}
  }
\end{enumerate}
\end{document}
