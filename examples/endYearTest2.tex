% %%%%%%%%%%%%%%%%%%%%%%%%%%%%%%%%%%%%%%%%%%%%%%%%%%%%%%%%%%%%%%%%%%%%%%
%                       Examples for the xelatex package schl
%                               Tassos Tsesmetzis
%
% %%%%%%%%%%%%%%%%%%%%%%%%%%%%%%%%%%%%%%%%%%%%%%%%%%%%%%%%%%%%%%%%%%%%%%
\documentclass[12pt,a4page]{article}
\usepackage{fontspec}
\usepackage{xunicode}
\usepackage{xltxtra}
\usepackage{xgreek}

\setmainfont[Mapping=tex-text]{Linux Biolinum O}

% Package geometry. Set text area dimensions.
\usepackage[width=0.9\paperwidth,height=0.9\paperheight,centering]{geometry}
% Package amsfonts. We need it for the macro \mathbb.
\usepackage{amsfonts}
% Package enumerate. 
\usepackage{enumerate}
% Package bm.
\usepackage{bm}
% Package tikz
\usepackage{tikz}
\usetikzlibrary{calc}
% Load package schl.
\usepackage[greek]{schl}
% Load the gr-math package
\usepackage{gr-math}

% Don't print page number
\pagestyle{empty}

% Set \school.
\school{ΓΕΛ Βόλου}
% Set name of the teacher.
\teacher{Όνομα εκπαιδευτικού}
% Set subject.
\subject{Μαθηματικά Κατεύθυνσης}
% Set grade.
\grade{Β' Λυκείου}
% Set \authorityii.
\authorityiii{%
  ΠΕΡΙΦ/ΚΗ Δ/ΝΣΗ  Α'/ΘΜΙΑΣ ΚΑΙ\\
  Β'/ΘΜΙΑΣ ΕΚΠ/ΣΗΣ ΒΟΛΟΥ\\
  Δ/ΝΣΗ Β'/ΘΜΙΑΣ ΕΚΠΑΙΔΕΥΣΗΣ\\
  ΒΟΛΟΥ
}
% Set school year.
\schoolyear{2018--19}
% Set test date.
\setdate{18/05/2019}
% Set examination time.
\examtime{8:30}

% Default number of dots for \lowerdots.
\def\defaultDots{10}

% Set vertical space between enumerate items.
\setlist*[enumerate]{itemsep=2ex}%
% Set vertical space between tasks.
\setlist*[schltask]{itemsep=2ex}%
% Print text in multichoice environment in \small size.
\setlist*[multichoice]{before=\small\hspace*{4em},label=\bf\alph*),itemjoin=\hspace{0.15\linewidth}}

\begin{document}
% Print \authoritylogo.
\authoritylogo[3]
\hfill%
\examdetailsii

\vspace{4ex}

% Set the header.
\finalexamhd{ΓΡΑΠΤΕΣ ΠΡΟΑΓΩΓΙΚΕΣ}{ΜΑΪΟΥ -- ΙΟΥΝΙΟΥ}

\vspace*{1.0ex}

\setlength \baselineskip{14.5pt}

\begin{schltask}
\item \leavevmode%
  \begin{enumerate}[label=\bf Α\arabic*.,before=\vspace{-2ex},leftmargin=24pt]
  \item Θεωρούμε σημείο αναφοράς $\bm{O}$ και ευθύγραμμο τμήμα $\bm{AB}$. Αν $\bm{M}$ είναι το μέσο του $\bm{AB}$, τότε, να αποδείξετε ότι\points{6}\phantom{.} % Crazy, but without \phantom macro \points leaks to next line
    $$\bm{\overrightarrow{OM} = \frac{\overrightarrow{OA} + \overrightarrow{OB}}{2}}$$
  \item Να χαρακτηρίσετε κάθε μία από τις παρακάτω προτάσεις ως {\textit σωστή} ή {\textit λάθος}:\points{9}
  \begin{enumerate}[label=\bf\roman*)]
  \item Το διάνυσμα $\vec{\gamma} = (3 \,,\, -2)$ είναι παράλληλο στην ευθεία $2x + 3y + 5 = 0$.
  \item Το συμμετρικό του σημείου $A(2 \,,\, 3)$ ως προς τον άξονα $x'x$ είναι το $B(2 \,,\, -3)$.
  \item  Τα διανύσματα $\vec{u} = (-3 \,,\, 4)$ και $\vec{v} = (2 \,,\, -1)$ είναι κάθετα.
  \end{enumerate}
\item Να επιλέξετε, τη σωστή απάντηση στις ακόλουθες προτάσεις: \points{10}
  \begin{enumerate}[label=\bf\roman*)]
  \item Μία ευθεία κάθετη στην ευθεία $y = -x$ είναι η\\
    \begin{multichoice}
    \item $y = 2x + 1$
    \item $y = x - 1$
    \item $y = -x +1$
    \end{multichoice}
  \item Το εσωτερικό γινόμενο δύο διανυσμάτων είναι\\
    \begin{multichoice}
    \item αριθμός
    \item ευθύγραμμο τμήμα
    \item διάνυσμα
    \end{multichoice}
  \end{enumerate}
\end{enumerate}
\item \points{25}\\[-1.5ex]
  \begin{minipage}[b]{210pt}
    \setlength \baselineskip{14.5pt}
    \begin{enumerate}[label=\bf Β\arabic*.,leftmargin=24pt]
    \item Να αποδείξετε ότι η εξίσωση της ευθείας $\varepsilon$ του διπλανού σχήματος είναι $$\varepsilon:\hspace{3pt} y = x - 2$$
    \item Να βρείτε το σημείο της $\varepsilon$ που έχει τη μικρότερη απόσταση από την αρχή των αξόνων $O$.
    \end{enumerate}
\end{minipage}\hspace{8em}
\begin{tikzpicture}
  [%
  axes/.style={->, >=stealth, thick},
  scale=0.5
  ]
  \draw [axes] (-4,0) -- (4,0) node [below] {\small$\bm{x}$};
  \draw [axes] (0,-4) -- (0,4) node [left] {\small$\bm{y}$};
  %
  \coordinate [label=above left:\footnotesize$O$] (O) at (0,0);
  \coordinate [label=below:\footnotesize$2$] (A) at (2,0);
  \coordinate [label=above:\footnotesize$\bm{\varepsilon}$] (B) at (-2, -4);
  \coordinate (C) at (4, 2);
  % 
  \fill (A) circle [radius=3pt];%
  %
  \draw [very thick] (B) -- (C);
  \draw (A) -- ($(0.4,0) + (A)$) arc [start angle=0, end angle=45, radius=0.4] -- cycle;
  %
  \node at ($(A)+(22:1.1)$) {\scriptsize$45^\circ$};
\end{tikzpicture}
\item \leavevmode\\
  Δίνεται η εξίσωση 
  $$2\lambda x + y - 1 = 0 \eqno(1)$$
  όπου $\lambda\in\mathbb{R}$.
  \begin{enumerate}[label=\bf Γ\arabic*.,leftmargin=24pt]
  \item Να αποδείξετε ότι για κάθε τιμή της παραμέτρου $\lambda$, η εξίσωση (1) παριστάνει ευθεία.\points{5}
  \item Να βρείτε το σημείο τομής των ευθειών που ορίζονται από την εξίσωση (1).\points{20}
  \end{enumerate}
\item \leavevmode\\
  Θεωρήστε την εξίσωση 
  $$x^2 + x y - 6 y^2 = 0 \eqno(2)$$
  \begin{enumerate}[label=\bf Δ\arabic*.,leftmargin=24pt]
  \item Να αποδείξετε ότι η εξίσωση (2) παριστάνει τις ευθείες\points{5}
    $$ \varepsilon:\hspace{3pt} x - 2 y = 0 \qquad \hbox{και} \qquad \delta:\hspace{3pt} x + 3y = 0 $$
  \item Να δείξετε ότι για κάθε σημείο M του άξονα $x'x$ ισχύει $d(M\,,\,\varepsilon) = \sqrt{2} \, d(M\,,\,\delta)$.\points{5}
  \item Ένας κύκλος $C$ έχει το κέντρο του $K$ στον θετικό ημιάξονα $Ox$, εφάπτεται στη μία από τις ευθείες $\varepsilon$ και $\delta$, και δεν έχει κανένα κοινό σημείο με την άλλη ευθεία. Αν η μεγαλύτερη από τις αποστάσεις του $K$ από τις $\varepsilon$, $\delta$ είναι  $\sqrt{5}$, τότε, να βρείτε την εξίσωση του κύκλου $C$.\points{10}
  \end{enumerate}
\end{schltask}

\vspace{8ex}

{% Change font for the wish.
  \setmainfont[Mapping=tex-text]{GFS Didot}%
  \wish%
}

\vspace{4ex}

\noindent{\bf Παρατήρηση:} \textit{Όλες οι απαντήσεις να γραφούν στην κόλλα σας.}


\vspace{7ex}

\signatures{\textit{Ονομ/νυμο διευθυντή}}
\hfill
\signatures[Οι Εισηγητές]{\textit{Ονομ/νυμο εισηγητή Α},\textit{Ονομ/νυμο εισηγητή Β}}
\end{document}
