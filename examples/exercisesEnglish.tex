% %%%%%%%%%%%%%%%%%%%%%%%%%%%%%%%%%%%%%%%%%%%%%%%%%%%%%%%%%%%%%%%%%%%%%%
%                       Examples for the xelatex package schl
%                               Tassos Tsesmetzis
%
% %%%%%%%%%%%%%%%%%%%%%%%%%%%%%%%%%%%%%%%%%%%%%%%%%%%%%%%%%%%%%%%%%%%%%%


\documentclass[12pt,a4page]{article}
\usepackage{xunicode}
\usepackage{xltxtra}
\usepackage{xgreek}

\setmainfont[Mapping=tex-text]{Linux Biolinum O}

% Package geometry. Set text area dimensions.
\usepackage[width=0.9\paperwidth,height=0.9\paperheight,centering]{geometry}
% Package amsfonts. We need it for the macro \mathbb.
\usepackage{amsfonts}
\usepackage{xcolor}

% Load the package schl
\usepackage[greek]{schl}

% Don't print page number
\pagestyle{empty}

% Set \school.
\school{Kronos High School}
% Set name of the teacher.
\teacher{\textit{teacher's name}}
% Set suject.
\subject{Algebra}
% Set grade.
\grade{7th grade}

\makeatletter
% Define some terms in English
\def\deadline@term{Deadline}
\def\hint@term{Hint}
\def\answerabbr@term{Ans.}
\makeatother

\setlist*[exercise]{
  label={%
    \parbox[t]{\linewidth}{
        \setlength \baselineskip{12pt}%
        \setmainfont[Mapping=tex-text]{Linux Libertine O}\color{blue}%
        \noindent\rule{\linewidth}{2pt}
        \newline\noindent\colorbox{blue!90!black!10}{\makebox[0.99\linewidth][l]{\textsc{Exercise} \arabic*}}%
        \\*[-1ex]\rule{\linewidth}{2pt}
    }
  }
} 

\newcommand\myitem[1]{%
  \colorbox{green!80!black!20}{\begin{minipage}{0.99\linewidth}#1\end{minipage}}%
}

\begin{document}
% Print school logo with size 40ex and no indentation.
\noindent\schoollogo{40ex}%
% Print deadline at the end of the current line.
\hfill\deadline{Monday, February 11}

% Leave some space between header and logo.
\vspace{4ex}

\begin{exercise}[itemsep=1ex]
\item \myitem{Calculate the value of the polynomial $P(x) = x^2 - x + 1$, when $x = 1$.\\
    % Print answer at the end of line. We need \phantom{.} because there is no
  % letter in front of \answer.
    \phantom{.}\answer{$1$}%
  }
\item \myitem{Solve the equation $\sin x = \cos x$.\\%
    \phantom{.}\answer{$x = \frac\pi{4} + \kappa \pi,\, \kappa\in\mathbb{Z}$}
    % Add a hint to the exercise.
    \hint{Observe that none of the numbers $x = \frac{\pi}2 + \kappa \pi,\, \kappa\in\mathbb{Z}$ is a root. \\
      For $x \neq \frac{\pi}2 + \kappa \pi,\, \kappa\in\mathbb{Z}$, we have
       $$\frac{\sin x}{\cos x} = 1 \Longleftrightarrow \tan x = 1
      \Longleftrightarrow \ldots$$
    }
  }
\item \myitem{Solve the equation $$x^3 - 4x^2 + x + 6 =0$$%
    \phantom{.}\answer{$x=-1$ or $x=2$ or $x=3$}
    \hint{Find a root $\rho$ among the integer divisors of the constant term $6$. Then devide: $(x^3 - 4x^2 + x + 6):(x - \rho)$ and factor the left side of the given equation.%
    }
  }
\item \myitem{Find the roots of the inequality
    $$e^{3x} \leq e^{2x + \ln 2} + e^x - 2$$
    \phantom{.}\answer{$0\leq x\leq \ln 2$}
    \hint{The inequality is written as:
      $$e^{3x} \leq 2 e^{2x} + e^x - 2 \Longleftrightarrow e^{3x} - 2 e^{2x} - e^x + 2 \leq 0$$
      Set $w = e^x$. Then, the inequality transforms in a polynomial inequality with variable $w$.
    }
  }
\item \myitem{For all $\lambda\in\mathbb{R}$, solve the system
    $$\left\{
      \begin{array}{cl}
        \lambda  x  - 2 y & = 1 \\
        4x  + y & = -2
      \end{array}
    \right.
    $$
    \phantom{.}\answer{%
      % Our answers splits in several lines. We use a \parbox to put all text in
      % a paragraph. First argument [c], selects as baseline of the text the horizontal
      % line through the midpoint of it's height. Second argument {150pt} is the
      % length of the box.
      \parbox[c]{150pt}{%
        % Choose the smallest size for our font.
        {\tiny
          \begin{itemize}
          \item If $\lambda \neq -8$ the system has the unique solution
            $(x\,,\, y) = (-\frac1{\lambda + 8}\,,\, -\frac{\lambda + 4}{\lambda + 8})$.
          \item If $\lambda = -8$ the system has no solution.
          \end{itemize}
        }
      }
    }%
    \hint{Calculate the determinants $D,\ D_x,\ D_y$. Then, find the values of
      $\lambda$ for which $D = 0$. We have two cases:
      \begin{itemize}[leftmargin=10pt]
      \item For all $\lambda\in\mathbb{R}$ for which $D \neq 0$, the system  has a unique solution\ldots
      \item For either one of the roots of the equation $D = 0$, the system either has infinite solutions, either has no solution\ldots          \end{itemize}
    }
  }
\end{exercise}
\end{document}