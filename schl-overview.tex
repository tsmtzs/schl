\documentclass[12pt,a4page]{article}
\usepackage{xunicode}
\usepackage{xltxtra}
\usepackage{xgreek}

\setmainfont[Mapping=tex-text]{Linux Biolinum O}

\usepackage[width=0.9\paperwidth,height=0.9\paperheight,centering]{geometry}
\usepackage{color}
\usepackage{amsfonts}
\usepackage{amsmath}
\usepackage{graphicx}
\usepackage{bm}
\usepackage{enumerate}
\usepackage{listings}

\lstset{
  basicstyle=\footnotesize\ttfamily,
  columns=flexible,
  breaklines=true,
  extendedchars=true
}

\pagestyle{empty}

% Load the package 'schl'.
\usepackage[english]{schl}

% Change some schl macro terms to English
% \makeatletter
% \def\exercise@term{Exercise}
% \def\task@term{Task}
% \makeatother

% Change the space between adjacent rows in a tabular environment.
\renewcommand{\arraystretch}{2.5}

\begin{document}

\heading{An overview of the \textit{schl} package}

\verb|schl| is a \XeLaTeX\, package that provides commands and environments suitable for document types that appear in a classroom enviromnent. It's development is based on the Greek school educational practice, but it may be usefull in other contexts also. This document offers a quick view of working examples for \verb|schl|'s marcos. If we load the package passing the parameter \verb|greek|, several macros will be printed in Greek. These are
defined in \verb|languages/sch-greek.def|. If you want to set them in a different language modify the \verb|languages/sch-template.def| file.

\begin{enumerate}
\item Blank space is designated with the macros \verb|\lowerdots| and \verb|\blankspace|.
  \begin{center}
    \footnotesize
    \begin{tabular}{c|p{0.4\textwidth}}
      \textit{\large code} & \hfill\textit{\large result}\hfill\phantom{.} \\ \hline
      % 1st example
      \begin{minipage}[t]{0.4\textwidth}
        \lstinline|Small spaces \lowerdots{3} and \blankspace{2em}.|
      \end{minipage} &
                       Small spaces \lowerdots{3} and \blankspace{2em}. \\
                       % 2nd example
      \begin{minipage}[t]{0.4\textwidth}
        \lstinline|Fill this bigger \lowerdots{20} space. And this one \blankspace{15em}.|
      \end{minipage} &
                       Fill this bigger \lowerdots{20} space.
                       And this one \blankspace{15em}. \\
                       % 3rd example
      \begin{minipage}[t]{0.4\textwidth}
        \lstinline|Change the vertical position \lowerdots[0.5ex]{10} and \blankspace[-2.0ex]{5em}.|
      \end{minipage} &
                       Change the vertical position \lowerdots[0.5ex]{10} and \blankspace[-2.0ex]{5em}.\\[2ex]
                       % 4rth example
      \begin{minipage}[t]{0.4\textwidth}
        \lstinline|Also in mathematical expressions $\cos\frac\pi4 = \lowerdots{4}$ and $\cos\frac\pi4 = \blankspace{2em}$.|
      \end{minipage} &
                       Also in mathematical expressions $\cos\frac\pi4 = \lowerdots{4}$ and $\cos\frac\pi4 = \blankspace{2em}$.
    \end{tabular}
  \end{center}
  % %%%%%%%%%%%%%%%%%%%%%%%%%%%%%%%%%%%%%%%%%%%%%%%%%%
  % Environment 'exercise'
  % %%%%%%%%%%%%%%%%%%%%%%%%%%%%%%%%%%%%%%%%%%%%%%%%%% 
\item With the environment \texttt{exercise} you can typeset exercises.
  \begin{center}
    \footnotesize
    \begin{tabular}{c|p{0.4\textwidth}}
      \textit{\large code} & \hfill\textit{\large result}\hfill\phantom{.} \\ \hline
      \begin{minipage}[t]{0.4\textwidth}
        % Remove white space from the start of every new line inside lstlisting environent
\begin{lstlisting}
\begin{exercise}
\item  Write all prime integers that are less or equall to $100$.
\item We 've bought $120$ watermelons from a local grocery shop. The total weight was $360\, kg$ and the watermelons were sold for $0.5\euro$ per $kg$. The grocer was highly delighted from this and decided to dedicate himself in the black art of Mathematics. Furthermore, he offered as a $2.5\%$ discount. How much money did we gave for the watermelons?
\item Prove that the sum of the angles of a triangle equals $180^\circ$.
\end{exercise}
        \end{lstlisting}
      \end{minipage} &
                       \begin{exercise}
                       \item  Write all prime integers that are less or equall to $100$.
                       \item We 've bought $120$ watermelons from a local grocery shop. The total weight was $360\, kg$ and the watermelons were sold for $0.5\euro$ per $kg$. The grocer was highly delighted from this and decided to dedicate himself in the black art of Mathematics. Furthermore, he offered as a $2.5\%$ discount. How much money did we gave for the watermelons?
                       \item Prove that the sum of the angles of a triangle equals $180^\circ$.
                       \end{exercise}
    \end{tabular}
  \end{center}
  % %%%%%%%%%%%%%%%%%%%%%%%%%%%%%%%%%%%%%%%%%%%%%%%%%% 
  % Environment schltask'
  % %%%%%%%%%%%%%%%%%%%%%%%%%%%%%%%%%%%%%%%%%%%%%%%%%%
\item The environment \texttt{schltask} can be used for summative tests.
  %
    \begin{center}
    \footnotesize
    \begin{tabular}{c|p{0.4\textwidth}}
      \textit{\large code} & \hfill\textit{\large result}\hfill\phantom{.} \\ \hline
      \begin{minipage}[t]{0.4\textwidth}
        % Remove white space from the start of every new line inside lstlisting environent
\begin{lstlisting}
\begin{schltask}
\item Solve the equation $x^2 - 3x + 2 = 0$.
\item Prove the Pythagorean theorem.
\item Prove that the medians of a triangle have a common point.
\end{schltask}
        \end{lstlisting}
      \end{minipage} &
                       \begin{schltask}
                       \item Solve the equation $x^2 - 3x + 2 = 0$.
                       \item Prove the Pythagorean theorem.
                       \item Prove that the medians of a triangle have a common point.
                       \end{schltask}
    \end{tabular}
  \end{center}
  % %%%%%%%%%%%%%%%%%%%%%%%%%%%%%%%%%%%%%%%%%%%%%%%%%% 
  % Macro \asnswer
  % %%%%%%%%%%%%%%%%%%%%%%%%%%%%%%%%%%%%%%%%%%%%%%%%%% 
\item The macro \verb/\answer/ is used to typeset the answer of an exercise.
    %
    \begin{center}
    \footnotesize
    \begin{tabular}{c|p{0.4\textwidth}}
      \textit{\large code} & \hfill\textit{\large result}\hfill\phantom{.} \\ \hline
      \begin{minipage}[t]{0.4\textwidth}
        % Remove white space from the start of every new line inside lstlisting environent
\begin{lstlisting}
\begin{exercise}
  \item Find the sum $1 + 1$.\answer[\hfill\footnotesize]{2}
\end{exercise}
\end{lstlisting}
\end{minipage} &
      \begin{minipage}[t]{0.4\textwidth}
        \begin{exercise}
  \item Find the sum $1 + 1$.\answer[\hfill\footnotesize]{2}
  \end{exercise}
  \end{minipage}
    \end{tabular}
  \end{center}
  % %%%%%%%%%%%%%%%%%%%%%%%%%%%%%%%%%%%%%%%%%%%%%%%%%% 
  % Macro \solution
  % %%%%%%%%%%%%%%%%%%%%%%%%%%%%%%%%%%%%%%%%%%%%%%%%%% 
\item With the macro \verb|\solution|, we write the solution of an exercise.
      %
  \begin{center}
    \footnotesize
    \begin{tabular}{c|p{0.4\textwidth}}
      \textit{\large code} & \hfill\textit{\large result}\hfill\phantom{.} \\ \hline
      \begin{minipage}[t]{0.4\textwidth}
        % Remove white space from the start of every new line inside lstlisting environent
\begin{lstlisting}
\begin{exercise}
\item Prove that there are infinite prime numbers.
  \solution{%
    Assume that there is a finite number of primes $p_1,\ldots,p_\nu$. Define the integer\ldots}
\end{exercise}
\end{lstlisting}
\end{minipage} &
                 \begin{exercise}
                 \item Prove that there are infite prime numbers.
                   \solution{%
                     Assume that there is a finite number if primes $p_1,\ldots,p_\nu$. Define the integer\ldots}
                 \end{exercise}
    \end{tabular}
  \end{center}
  % %%%%%%%%%%%%%%%%%%%%%%%%%%%%%%%%%%%%%%%%%%%%%%%%%% 
  % Macro \points
  % %%%%%%%%%%%%%%%%%%%%%%%%%%%%%%%%%%%%%%%%%%%%%%%%%% 
\item Set points to exercises with the macro \verb|\points|:
        %
  \begin{center}
    \footnotesize
    \begin{tabular}{c|p{0.4\textwidth}}
      \textit{\large code} & \hfill\textit{\large result}\hfill\phantom{.} \\ \hline
      \begin{minipage}[t]{0.4\textwidth}
        % Remove white space from the start of every new line inside lstlisting environent
\begin{lstlisting}
\begin{schltask}
\item \points{25}\par
  Prove the theorem of Bolzano.
\item \points{11}\par
  Let $f:\mathbb{R}\rightarrow\mathbb{R}$ be a function with $f(x) = \frac1{x-1}$.
  \begin{enumerate}
  \item \points[\itshape]{10} Find its domain.
  \item \points[\itshape]{1} Calculate the value $f(3)$.
  \end{enumerate}
\end{schltask}
\end{lstlisting}
\end{minipage} &
                 \begin{schltask}
                 \item \points{25}\par
                   Prove the theorem of Bolzano.
                 \item \points{11}\par
                   Let $f:\mathbb{R}\rightarrow\mathbb{R}$ be a function with $f(x) = \frac1{x-1}$.
                   \begin{enumerate}
                   \item \points[\itshape]{10} Find its domain.
                   \item \points[\itshape]{1} Calculate the value $f(3)$.
                   \end{enumerate}
                 \end{schltask}
    \end{tabular}
  \end{center}
  % %%%%%%%%%%%%%%%%%%%%%%%%%%%%%%%%%%%%%%%%%%%%%%%%%%
  % Environment 'question'
  % %%%%%%%%%%%%%%%%%%%%%%%%%%%%%%%%%%%%%%%%%%%%%%%%%%
\item Environment \verb|question|:
        %
  \begin{center}
    \footnotesize
    \begin{tabular}{c|p{0.4\textwidth}}
      \textit{\large code} & \hfill\textit{\large result}\hfill\phantom{.} \\ \hline
      \begin{minipage}[t]{0.4\textwidth}
        % Remove white space from the start of every new line inside lstlisting environent
\begin{lstlisting}
\begin{question}
 \item Is there a bigger real number?
 \item Is there a smallest positive real number?
\end{question}
\end{lstlisting}
\end{minipage} &
                 \begin{question}
                 \item Is there a bigger real number?
                 \item Is there a smallest positive real number?
                 \end{question}
    \end{tabular}
  \end{center}
  % %%%%%%%%%%%%%%%%%%%%%%%%%%%%%%%%%%%%%%%%%%%%%%%%%%
  % Macro \hint
  % %%%%%%%%%%%%%%%%%%%%%%%%%%%%%%%%%%%%%%%%%%%%%%%%%%
\item Hints with the macro \verb|\hint|:
        %
  \begin{center}
    \footnotesize
    \begin{tabular}{c|p{0.4\textwidth}}
      \textit{\large code} & \hfill\textit{\large result}\hfill\phantom{.} \\ \hline
      \begin{minipage}[t]{0.4\textwidth}
        % Remove white space from the start of every new line inside lstlisting environent
\begin{lstlisting}
\begin{exercise}
\item Prove that between two rational numbers there is an irrational.
  \hint[\par\noindent\scriptsize]{%
    Assume rationals $\rho_1 < \rho_2$. We define the real number $\frac{\rho_1 + \rho_2}2$. Then, $x$ is\ldots}
\item Prove that $(\alpha + \beta)^2 = \alpha^2 + 2 \alpha \beta + \beta^2$.
  \hint[\par\noindent\scriptsize]{%
    We have $(\alpha + \beta)^2 = (\alpha + \beta) \cdot (\alpha + \beta) = \ldots$}
\end{exercise}
\end{lstlisting}
\end{minipage} &
                 \begin{exercise}
                 \item Prove that between two rational numbers there is an irrational.
                   \hint[\par\noindent\scriptsize]{%
                     Assume rationals $\rho_1 < \rho_2$. We define the real number $\frac{\rho_1 + \rho_2}2$. Then, $x$ is\ldots}
                 \item Prove that $(\alpha + \beta)^2 = \alpha^2 + 2 \alpha \beta + \beta^2$.
                   \hint[\par\noindent\scriptsize]{%
                     We have $(\alpha + \beta)^2 = (\alpha + \beta) \cdot (\alpha + \beta) = \ldots$}
                 \end{exercise}
    \end{tabular}
  \end{center}
\item Environment \texttt{multichoice} is for multiple choice questions:
  % leave a blank line
  
  \begin{multichoice}
  \item choice 1
  \item choice 2
  \end{multichoice}
  % leave a blank line

  Another example
  % leave a blank line

  \begin{multichoice}
  \item choice 1
  \item choice 2
  \item choice 3
  \end{multichoice}

  Or

  \begin{multichoice}[before=\hspace{3em},itemjoin=\hspace{3em},label=\bf\arabic*{})]
  \item  this is a very long choice 1
  \item this is an even longer choice 2\\\hspace*{9em}
  \item this is a remarkably long choice 3
  \end{multichoice}
\item Environment \texttt{tickchoice}. Horizontal

  \begin{tickchoice*}
  \item choice A
  \item choice B
  \item choice C
  \end{tickchoice*}

  and vertical
  
  \begin{tickchoice}
  \item choice A
  \item choice B
  \item choice C
  \end{tickchoice}
\item A wish for good luck
  \wish
  Setting the text. Macro \verb/\letterspace/ sets the space between adjucent letters
  \makeatletter
  \def\schl@wish{\letterspace{10} ΚΑΛΗ ΤΥΧΗ}
  \makeatother
  \wish
\item Write the name and date:\\
  \fullname\\
  \datefield \\[1ex]
  Also, with dots or a line for blank space:\\
  \fullname{\lowerdots{40}}\\ \datefield{\blankspace{10em}}\\[1ex]
  We could use \\
  \setdate{28 Μαΐου 2020}
  \datefield{\getdate}
\item Exercise deadline:\\
  \deadline{2/2/2058}\\
\item Set the duration of a test:\\ \duration{10'} or \duration[\it]{10'} or \duration[\rm]{10'}
\item Add a remark in a document:\\
  \remark{Αυτή είναι μια παρατήρηση.}\\
  \remark[\rm]{Αυτή είναι μια παρατήρηση.}\\
  \remark[\it]{Αυτή είναι μια παρατήρηση.}
\item Add a reminder in a document:\\
  \reminder{Εδώ ξεκινά μια υπενθύμιση.}\\
  \reminder[\mdseries]{Εδώ ξεκινά μια υπενθύμιση.}
\item Header for the theory part of a test: \theorypart
  Header for the exercise part of a test: \exercisepart
\item Set the title of a worksheet
  \worksheethd{}
  or
  \worksheethd{στην παράγραφo \S A.2.3}
\item Teacher/headmaster signatures:\\
  \signatures{Georg Cantor}
  \hfill
  \signatures[Οι Εισηγητές]{Αλφαβήτας Γαμαδέλτας,Εψιλονζήτας Ηταθήτας}
\item Headers for tests:
  \examhd{}
  \examhd{Α' τετραμήνου}
  \examhd[Τεστ]{στο κεφάλαιο 1}
\item Header for end year summative tests:
  \finalexamhd{ΓΡΑΠΤΕΣ ΕΠΑΝΑΛΗΠΤΙΚΕΣ}{ΜΑΪΟΥ -- ΙΟΥΝΙΟΥ}
\item Logo of the  exams

  % \authoritylogo\\
  
  or if we set \verb+\authorities+ and \verb+schl@authorities+:

  \school{%
    ΕΛΛΗΝΙΚΗ ΔΗΜΟΚΡΑΤΙΑ

    \vspace{3\lineskip}

    ΥΠΟΥΡΓΕΙΟ ΠΑΙΔΕΙΑΣ, ΕΡΕΥΝΑΣ \& ΘΡΗΣΚΕΥΜΑΤΩΝ

    \vspace{3\lineskip}

    ΔΙΕYΘΥΝΣΗ Β/ΘΜΙΑΣ ΕΚΠΑΙΔΕΥΣΗΣ ΜΕΣΣΗΝΙΑΣ
  }
  \school{Λύκειο Καλαμάτας}
  % \authoritylogo
\item School logo

  \school{ΓΥΜΝΑΣΙΟ ΠΑΤΡΩΝ}
  \grade{Β' Γυμνασίου}
  \subject{Μαθηματικά}
  \teacher{Ήρων από την Αλεξάνδρεια}
  \schoollogo{200pt}
\item True-false type questions with the environment \verb|truefalse|
  \begin{truefalse}
  \item kjahs naoisjh nmaksjnd njaksjn dnamksdoh n ash nda
    ias doasj d jjsn ndijewh nasusfd has hujh djnjdi haiusd i
  \item kjahs naoisjh nmaksjnd njaksjn dnamksdoh n ash nda
    ias doasj d jjsn ndijewh nasusfd has hujh djnjdi haiusd i
    kjahs naoisjh nmaksjnd njaksjn dnamksdoh n ash nda
    ias doasj d jjsn ndijewh nasusfd has hujh djnjdi haiusd i
    kjahs naoisjh nmaksjnd njaksjn dnamksdoh n ash nda
  \item  ias doasj d jjsn ndijewh nasusfd has hujh djnjdi haiusd i
  \item     ias doasj d jjsn ndijewh nasusfd has hujh djnjdi haiusd i
    kjahs naoisjh nmaksjnd njaksjn dnamksdoh n ash nda
    ias doasj d jjsn ndijewh nasusfd has hujh djnjdi haiusd i
    kjahs naoisjh nmaksjnd njaksjn dnamksdoh n ash nda
  \end{truefalse}
\item \verb|truefalse*| is a variant of \verb|truefalse|.

  \begin{truefalse*}
  \item kjahs naoisjh nmaksjnd njaksjn dnamksdoh n ash nda
    ias doasj d jjsn ndijewh nasusfd has hujh djnjdi haiusd i
  \item kjahs naoisjh nmaksjnd njaksjn dnamksdoh n ash nda
    ias doasj d jjsn ndijewh nasusfd has hujh djnjdi haiusd i
    kjahs naoisjh nmaksjnd njaksjn dnamksdoh n ash nda
    ias doasj d jjsn ndijewh nasusfd has hujh djnjdi haiusd i
    kjahs naoisjh nmaksjnd njaksjn dnamksdoh n ash nda
  \item  ias doasj d jjsn ndijewh nasusfd has hujh djnjdi haiusd i
  \item     ias doasj d jjsn ndijewh nasusfd has hujh djnjdi haiusd i
    kjahs naoisjh nmaksjnd njaksjn dnamksdoh n ash nda
    ias doasj d jjsn ndijewh nasusfd has hujh djnjdi haiusd i
    kjahs naoisjh nmaksjnd njaksjn dnamksdoh n ash nda
  \end{truefalse*}
\item Matching questions:
  \setlist*[leftmatching]{label=}
  \setlist*[rightmatching]{label=}
  \setlength{\rightmatchwidth}{200pt}
  \matchingque[320pt]{παιδί,χταπόδι,παιχνίδι}{θάλασσα,κατάστημα,διάστημα,διάβασμα,ψωμί, σαλάμι}
\end{enumerate}
\end{document}